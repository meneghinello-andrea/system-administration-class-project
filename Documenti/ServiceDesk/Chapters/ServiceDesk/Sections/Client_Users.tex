%------------------------------------------------
%
% Client_Users.tex 
%
% This section illustrates the client and the
% users of the service desk
%------------------------------------------------
\section[Clienti ed utenti]{clienti ed utenti}
\label{sd-client-users}
In ambiente \ac{Information-Technology-Infrastructure-Library} \keyword{cliente} ed \keyword{utente} assumono un significato differente anche se talvolta sono la stessa persona.

Per \ac{Information-Technology-Infrastructure-Library} il cliente è colui che acquista e quindi possiede i servizi \acs{Information-Technology}. In questo caso il ruolo è assunto dagli amministratori/amministratori delegati dell'istituto.

Gli utenti sono invece coloro che utilizzano i servizi \acs{Information-Technology} e quindi sono quelli che sono più affetti da un eventuale blocco degli stessi. In questo caso gli utenti del \english{Service Desk} sono i medici egli infermieri che operano all'interno dell'istituto utilizzando i servizi \acs{Information-Technology}.

\subsection[Esperienza utente]{esperienza utente}
\label{sd-users-experience}
L'obiettivo di questa parte del documento è quella di fornire alcune semplici linee guida (\english{best practies}) che mirano a rendere il \english{Service Desk} un'entità vicina alle esigenze degli utenti e di facile utilizzo, con la conseguenza di avere una buona percezione dagli utenti.

Migliorare l'esperienza dell'utente finale, non si limita a beneficio dell'utente ma aiuta nell'erogazione dei servizi e rafforza molto di più questa funzionalità.

Alcuni suggerimenti per migliorare l'esperienza utente sono:

\begin{itemize}
\item{fornire il materiale necessario allo staff tecnico;}
\item{rendere le informazioni necessarie sempre disponibili;}
\item{offrire agli utenti differenti canali di comunicazione;}
\item{offrire agli utenti \english{software} \acs{Software-as-a-Service};}
\item{utilizzare il \english{desktop} remoto per risoluzioni veloci;}
\item{aiutare gli utenti ad aiutarsi.}
\end{itemize}

\subsubsection[Fornire il materiale necessario]{fornire il materiale necessario}
Fornendo allo staff tecnico del \english{Service Desk} il materiale di cui necessita per operare, esso aumenterà di conseguenza la propria produttività. Come risultato si ottiene una funzionalità che è molto veloce nel fornire aiuto, apparendo cosi versatile agli occhi degli utenti.

Un \english{Service Desk} funzionale ha bisogno di strumenti per gestire correttamente le richieste, la priorità, i report ed il mantenimento dei registri. Esistono degli strumenti informatici che \keyword{agevolano} e \keyword{velocizzano} queste fasi di lavoro all'interno del \english{Service Desk}.

\subsubsection[Accessibilità delle informazioni]{accessibilità delle informazioni}
Lo staff tecnico del \english{Service Desk} deve avere facile accesso alle informazioni sui problemi noti, le relative soluzioni, le imminenti modifiche all'infrastruttura \acs{Information-Technology} e quali strumenti/servizi sono assegnati ad un particolare utente.

Attraverso il facile accesso alle informazioni precedenti qualunque membro dello staff saprà come gestire al meglio una richiesta.

Questo comporta un aumento della velocità di risposta del \english{Service Desk} con conseguente miglioramento della percezione da parte dell'utilizzatore finale.

\subsubsection[Canali di comunicazione]{canali di comunicazione}
La fornitura di differenti canali di comunicazione (vedi sezione \ref{sd-contact-mode}) agli utenti finali rende la funzione di \english{Service Desk} maggiormente accessibile a diverse categorie di utenti.

Vista però dal punto di vista del \english{Service Desk} questo potrebbe generare confusione e rallentare il lavoro.

Attraverso però l'uso di strumenti appropriati e automatici è possibile far si che le richieste entranti da differenti canali di comunicazione siano automaticamente reindirizzate in un unico canale visibile al personale del \english{Service Desk}.

Ad esempio l'utilizzo di un \english{software} che converta automaticamente le e-mails ricevute su uno specifico indirizzo in richieste di assistenza/aiuto.

\subsubsection[Fornitura di software SaaS]{fornitura di software SaaS}
Se la funzione di \english{Service Desk} utilizza come \english{software} di supporto, alle proprie mansioni, un \ac{Software-as-a-Service} l'intero dipartimento \acs{Information-Technology} ha una preoccupazione in meno quando si verificano manutenzioni \english{hardware}, installazioni di \english{\glossarySingolarTerm{patch}}, \english{fix} di sicurezza e aggiornamenti \english{software}.

Il fornitore (interno/esterno) del servizio di \ac{Software-as-a-Service} di \english{Service Desk} assicura che ogni membro dello staff tecnico lavorerà con la versione corretta dello stesso, e che aggiornamenti, \english{pathces}, e risoluzione di problemi siano installati automaticamente.

\subsubsection[Utilizzo della funzione di desktop remoto]{utilizzo della funzione di desktop remoto}
Alcuni problemi possono essere risolti attraverso l'uso della funzionalità di \english{Desktop} remoto. Cosi che i membri dello staff tecnico non devano raggiungere, ogni volta, la postazione di lavoro dell'utente.

Il tempo risparmiato può essere significativo quando lo staff tecnico riesce a risolvere il problema dalla propria postazione di lavoro, e l'utente può tornare ad essere produttivo con il più piccolo \english{downtime} possibile.

\subsubsection[Aiutare gli utenti ad aiutarsi]{aiutare gli utenti ad aiutarsi}
Ci sono molteplici modalità in cui un utente può aiutare se stesso, ovvero risolvendo le proprie difficoltà senza ricorre in un aiuto attivo da parte dello staff tecnico del \english{Service Desk}.

Una possibile modalità consiste nel corredare il portale, con cui si ricevono le richieste dagli utenti, con una sezione in cui sono elencati gli \glossaryPluralTerm{errore} più comuni. Come ad esempio la procedura di cambio \english{password}, la stampa attraverso una stampante di rete, ecc..

Se richieste minori possono essere risolte autonomamente dagli utenti faranno risparmiare tempo allo staff tecnico del \english{Service Desk}, consentendogli di dedicarsi alla risoluzione di incidenti più gravi.