%------------------------------------------------
%
% Input_output.tex 
%
% This section introduces the incident management
% process.
%------------------------------------------------
\section[Input e output del processo]{input e output del processo}
\label{im-io}
In questa sezione vengono illustrati gli \english{input} e gli \english{output} del processo, specificando da chi sono ricevuti e a chi sono rivolti.

\subsection[Input del processo]{input del processo}
\label{im-io-input}
In Tabella \ref{im-io-input-table} sono elencati gli \english{input} del processo e da chi sono ricevuti.

\begin{center}
\begin{longtable}{| p{6cm} | p{7cm} |}
\caption{Input del processo}
\label{im-io-input-table}\\
\hline
\multicolumn{1}{| c |}{\textbf{Input}} & \multicolumn{1}{| c |}{\textbf{Da}}\\
\hline
\endfirsthead
\hline
\multicolumn{1}{| c |}{\textbf{Input}} & \multicolumn{1}{| c |}{\textbf{Da}}\\
\hline
\endhead
Alimentazione elettrica & \ac{Uninterruptible-Power-Supply}\\
\hline
Traffico di rete rete & \english{Firewall}\\
\hline
Traffico sui dati & \ac{DataBase-Management-System}\\
\hline
\english{Log} applicativi & Applicazioni\\
\hline
Notifiche accessi/uscite & Sistemi di controllo accessi\\
\hline
Notifiche di emergenza & Sensori antifumo - antincendio\\
\hline
\end{longtable}
\end{center}

\subsection[Output del processo]{output del processo}
\label{im-io-output}
Non tutti gli eventi che giungono al processo generano \english{output}, ma sono comunque tutti tracciati e catalogati.

\begin{center}
\begin{longtable}{| p{6cm} | p{7cm} |}
\caption{output del processo}
\label{im-io-output-table}\\
\hline
\multicolumn{1}{| c |}{\textbf{Input}} & \multicolumn{1}{| c |}{\textbf{Da}}\\
\hline
\endfirsthead
\hline
\multicolumn{1}{| c |}{\textbf{Input}} & \multicolumn{1}{| c |}{\textbf{Da}}\\
\hline
\endhead
\english{ticket} & \english{Service Desk}\\
\hline
\end{longtable}
\end{center}

Gli eventi che generano sempre un \english{output} sono i \english{warning} ed \english{exception}. Le due tipologie generano \english{ticket} con priorità differenti, i \english{warning} generano \english{ticket} con priorità \attribute{2 - media} mentre le \english{exception} generano \english{ticket} con priorità \attribute{1 - alta} (vedi Sezione \ref{im-management-priority}).