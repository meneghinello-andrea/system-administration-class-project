%------------------------------------------------
%
% Roles-Responsability.tex 
%
% This section illustrates the roles and the
% responsability present in a service desk.
%------------------------------------------------
\section[Ruoli e responsabilità]{ruoli e responsabilità}
\label{sd-roles-responsabilities}
La chiave per un \english{Service Desk} \keyword{efficace} ed \keyword{efficiente} è quella di garantire che vi sia una chiara responsabilità e che i ruoli siano chiaramente definiti in modo da portare a termine la pratica di servizio. 

Un ruolo è strettamente legato alla descrizione di un lavoro oppure di un lavoro di gruppo ma non necessariamente ha bisogno di essere svolto da un singolo individuo.

La taglia dell'organizzazione, come essa è strutturata, l'esistenza di \english{patners} esterni ed altri fattori influenzeranno come i ruoli saranno assegnati. Tuttavia un particolare ruolo può essere svolto da una singolo individuo oppure condiviso da due o più persone, l'importante è la consistenza e la responsabilità d'esecuzione, assieme all'interazione con gli altri ruoli presenti nel dipartimento.

I seguenti ruoli sono necessari all'interno di un \english{Service Desk}:

\begin{itemize}
\item{Service Desk Manager}
\item{Service Desk Supervisor}
\item{Service Desk Analyst}
\end{itemize}

\subsection[Service Desk Manager]{service desk manager}
\label{sd-sd-manger}
In un'organizzazione come l'Ospedale Gaetano Pini dove sono presenti contemporaneamente un numero rilevante di servizi \acs{Information-Technology}, il ruolo di \english{Service Desk Manager} è giustificato.

Si avranno poi dei supervisori di aree di competenza specifiche che fanno direttamente a capo a lui/lei.

Il \english{Service Desk Manager} prende la responsabilità per le seguenti attività:

\begin{itemize}
\item{gestione di tutte le attività del \english{Service Desk} inclusa la supervisione;}
\item{agire come ulteriore punto di \english{escalation} per i supervisori;}
\item{svolgere un più ampio ruolo di servizio per gli utenti/clienti;}
\item{riportare alla sezione di \english{Service Strategy} ogni richiesta che può significativamente impattare il \english{business};}
\item{partecipare alle sedute del \ac{Change-Advisory-Board};}
\item{Assumersi la responsabilità globale per la gestione degli incidenti e la richiesta di realizzazione del \english{Service Desk}. Ciò può essere esteso a qualsiasi altra attività assunta dal \english{Service Desk}.}
\end{itemize}

In tutti i casi, descrizioni dei lavori devono essere ben definite redatti ed approvati affinché le responsabilità specifiche siano note.

\subsection[Service Desk Supervisor]{service desk supervisor}
\label{sd-sd-supervisor}
All'interno del \english{Service Desk} vi saranno differenti \english{Service Desk Supervisor} che avranno il compito di supervisionare specifiche aree dell'ambiente \acs{Information-Technology}.

Il personale dello staff che assume questo ruolo prende la responsabilità per le seguenti attività:

\begin{itemize}
\item{stabilire turni di lavoro affinché il livello di servizio garantito dal personale sia mantenuto durante l'orario operativo;}
\item{intraprendere attività di \ac{Human-Resources} in base alle esigenze;}
\item{produzione di statistiche e report per l'area di competenza;}
\item{rappresentare l'area di competenza durante le riunioni con il \english{Service Desk Manager};}
\item{organizzare la formazione del personale;}
\item{collegamento con il processo di \ac{Change-Management}}
\item{effettuare dei \english{briefings} con lo staff in merito a cambiamenti o sviluppi che possono influenzare il volume delle richieste presso il \english{Service Desk};}
\item{Assistere gli analisti durante la prima linea di supporto quando i carichi di lavoro sono elevati, o dove la richiesta richiede maggior esperienza.}
\end{itemize}

Un singolo individuo può essere il responsabile di più aree di competenza.

\subsection[Service Desk Analyst]{service desk analyst}
\label{sd-sd-analyst}
Il ruolo primario di un \english{Service Desk Analyst} è quello di fornire un supporto di primo livello attraverso la ricezione delle chiamate e successiva gestione degli incidenti che ne conseguono oppure la gestione delle richieste di servizio oppure il monitoraggio dell'ambiente \acs{Information-Technology}.

Le sue attività avvengono all'interno dei processi di \acf{Incident-Management}, \acf{Request-Fulfillment} ed \acf{Event-Management} in linea con gli obiettivi del \english{Service Desk}.