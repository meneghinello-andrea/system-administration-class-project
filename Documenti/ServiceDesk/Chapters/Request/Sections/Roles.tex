%------------------------------------------------
%
% Roles.tex 
%
% This section introduces the roles
% of the request fulfillment process.
%------------------------------------------------
\section[Ruoli e responsabilità]{Ruoli e responsabilità}
\label{rf-roles}
Dato che le richieste che possono giungere al processo possono riguardare ambienti molto differenti il proponente ritiene corretta la seguente suddivisione in ruoli all'interno del processo.

\subsection[Service Provider Group Manager]{service provider group manager}
\label{rf-roles-spgm}
Esiste un unica figura all'interno di questo processo con questo specifico ruolo che possiede quindi le seguenti responsabilità:

\begin{itemize}
\item{sviluppare la lista delle più complesse e più comuni richieste di servizio;}
\item{monitorare la coda delle richieste ed assicurarsi che tutte vengano gestite;}
\item{prendere la decisione finale sulle priorità delle richieste;}
\item{prendere decisioni sulle risorse necessarie per il compimento di una richiesta;}
\item{effettuare revisioni sulla lista delle richieste al fine di garantire il rispetto delle politiche, le attività ed eventuali vulnerabilità o rischi;}
\item{eseguire un esame post-attuazione al fine di garantire che tutti i sistemi funzionino e che la documentazione sia corretta;}
\item{rivedere le attività di processo qualora non rispettassero gli \ac{Service-Level-Agreement}.}
\end{itemize}

\subsection[Service Provider Group]{service provider group}
\label{rf-roles-spg}
I gruppi di servizio sono composti da un supervisore e personale di servizio o supporto tecnico. Essi possiedono le seguenti responsabilità:

\begin{itemize}
\item{specializzare le attività di processo per le specifiche richieste che devono soddisfare;}
\item{eseguire le attività qualora la richiesta si presenti;}
\item{in coordinamento con il \english{Service Desk} e sotto le loro linee guida, eseguire test post-attuazione al fine di determinare che tutti i servizi funzionino e che la documentazione sia completa.}
\end{itemize}