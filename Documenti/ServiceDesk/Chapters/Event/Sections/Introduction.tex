%------------------------------------------------
%
% Introduction.tex 
%
% This section introduces the request fulfillment
% process.
%------------------------------------------------
\section[Introduzione]{introduzione}
\label{rf-introduction}
Questo capitolo descrive il processo di \acf{Event-Management} per la funzione di \english{Service Desk} dell'\entity{}.

Il processo fornisce un metodo coerente che gli utenti del \english{Service Desk} dovranno utilizzare per monitorare l'ambiente \acs{Information-Technology} in cui sono ospitati i servizi offerti.

\subsection[Scopo principale]{scopo principale}
\label{em-introduction-scope}
Lo scopo principale che questo processo possiede, e per questo si rende necessario implementarlo all'interno del nuovo \english{Service Desk}, è quello di porsi come base per il \keyword{monitoraggio} ed il \keyword{controllo} dell'ambiente \acs{Information-Technology}.

Un funzionamento \english{efficace} di un qualsiasi servizio \acs{Information-Technology} dipende dalla conoscenza dello stato dell'infrastruttura e dal rilevare qualsiasi deviazione dal funzionamento atteso. Questo lo si ottiene attraverso un monitoraggio continuo e l'adozione di sistemi di controllo.

Inoltre un'opportuna programmazione degli eventi il personale del \english{Service Desk} è in grado di effettuare azioni di ripristino ancor prima che l'incidente possa verificarsi.

\subsection[Obiettivi]{obiettivi}
\label{em-introduction-objectives}
Instanziando il processo si intendono raggiungere i seguenti \keyword{obiettivi}:

\begin{itemize}
\item{controllare l'evoluzione dei \ac{Configuration-Item} presenti;}
\item{controllare le condizioni ambientali (entrate/uscite - antincendio);}
\item{monitorare l'utilizzo delle licenze \english{software}.}
\end{itemize}

\subsection[Definizioni]{definizioni}
\label{em-introduction-definitions}

\subsubsection{evento}
Un \keyword{evento} può essere definito come qualsiasi occorrenza visibile o rilevabile che ha significato per la gestione dell'infrastruttura \acs{Information-Technology} e valutazione sull'impatto che potrebbe causare un disservizio.

Gli eventi sono in genere notifiche create da un servizio \acs{Information-Technology}, un elemento di configurazione (\acs{Configuration-Item}) oppure da uno strumento di monitoraggio.

\begin{itemize}
\item{\textbf{attivi}: sono sistemi che periodicamente interrogano lo stato degli elementi di configurazione (\ac{Configuration-Item}) del sistema. (tecnica di \english{polling});}
\item{\textbf{passivi}: sono sistemi che vengono notificati dai singoli elementi di configurazione (\ac{Configuration-Item}) del sistema quando un evento significativo accade.}
\end{itemize}

\subsubsection{evento informativo}
Si tratta di un particolare tipo di evento che non richiede azioni attive da parte degli operatori e non rappresenta casi eccezionali. E' generalmente memorizzato all'interno del dispositivo che lo ha generato oppure in un \english{database} logico per un determinato periodo temporale.

\subsubsection{evento ``warning''}
Un evento della tipologia ``\english{warning}'' è generato quando un servizio oppure un dispositivo sta per raggiungere una soglia critica. Intendono notificare al personale del \english{Service Desk} che vi è necessità di un controllare la situazione e prendere le dovute contromisure prima che venga generato un evento di \english{exception}.

\subsubsection{evento ``exception''}
Un evento della tipologia ``\english{exception}'' è generato quando un servizio o sistema sta operando in modo anomalo. Tipicamente significa che vi è una violazione nello \ac{Service-Level-Agreement} ed il \english{business} è stato impattato.

Generalmente rappresentano un totale fallimento del sistema, alterazione di funzionalità o degrado delle \english{performance}.