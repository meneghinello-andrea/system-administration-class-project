%------------------------------------------------
%
% Target.tex 
%
% This section introduces the request fulfillment
% target.
%------------------------------------------------
\section[Tempistiche]{tempistiche}
\label{rf-target}
Alle richieste di servizio sono associate due variabili chiave, utilizzate al fine di raggiungere un servizio veloce ed efficiente:

\begin{itemize}
\item{\textbf{risposta}: tempo necessario affinché la richiesta si assegnata ad un operatore;}
\item{\textbf{risoluzione}: tempo necessario per completare la richiesta dopo che è stata accettata.}
\end{itemize}

\subsection[Richieste di assistenza]{richiesta di assistenza}
\label{rf-target-asr}
Per loro natura le richieste a servizi \english{standard} sono di \english{routine}. Gli obiettivi posso variare in base al tipo specifico di richiesta ed inoltre questi obiettivi possono cambiare con il miglioramento delle tecnologie a supporto del processo.

Gli obiettivi sono i seguenti:

\begin{itemize}
\item{\textbf{risposta}: 1 giornata lavorativa;}
\item{\textbf{risoluzione}: 5 giornate lavorative.}
\end{itemize}

\subsection[Richieste di modifica/miglioramento]{richieste di modifica/miglioramento}
\label{rf-target-ecr}
Le richieste di modifica/miglioramento sono univoche e quindi richiedono delle analisi costo/beneficio prima di essere approvate. Conseguentemente non è possibile definire a priori tempi in cui rispettarle.