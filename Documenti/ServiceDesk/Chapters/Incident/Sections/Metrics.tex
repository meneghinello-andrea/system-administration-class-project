%------------------------------------------------
%
% Metrics.tex 
%
% This section introduces the incident management
% metrics.
%------------------------------------------------
\section[Metriche di processo]{metriche di processo}
\label{im-metrics}
Il processo di \ac{Incident-Management} produce mensilmente dei \english{report} che illustrano il lavoro svolto durante la mensilità. All'interno di tali report sono riportate alcune metriche che indicano la qualità di processo e di conseguenza dei servizi offerti.

Le metriche utilizzate sono le seguenti:

\begin{itemize}
\item{numero totale di incidenti riscontrati;}
\item{numero di incidenti per ogni stato (es. registrato, in lavorazione, chiuso, ecc..);}
\item{numero di incidenti arretrati;}
\item{numero e percentuale di incidenti gravi;}
\item{tempo medio per raggiungere la risoluzione oppure un \english{\glossarySingolarTerm{workaround}}, riparti per codice d'impatto;}
\item{percentuale di incidenti che ha avuto la risoluzione entro i tempi concordati nello \ac{Service-Level-Agreement};}
\item{numero di incidenti riaperti dagli utenti e percentuale sul totale;}
\item{numero e percentuale di incidenti non assegnati in modo corretto;}
\item{numero e percentuale di incidenti non classificati in modo corretto;}
\item{percentuale di incidenti chiusi dal \english{Service Desk} senza necessità di scalare ad ulteriori livelli di supporto;}
\item{numero e percentuale di incidenti elaborati per ogni singolo utente dello staff;}
\item{numero e percentuale di incidenti risolti in remoto;}
\item{ripartizione degli incidenti sulle ore del giorno.}
\end{itemize}