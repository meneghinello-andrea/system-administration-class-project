%------------------------------------------------
%
% Abstract.tex 
%
% This section introduce the document.
%------------------------------------------------
\chapter[Abstract]{abstract}
\label{abs}
All'interno di una realtà aziendale complessa, come l'ente oggetto di questo capitolato, è doveroso istituire ed istruire dei processi e delle funzioni di controllo che mirano a garantire che i servizi offerti agli utenti siano il più \keyword{efficaci} ed \keyword{efficienti} possibile.

Un possibile metodo per ottenere degli ottimi livelli di servizi offerti e allo stesso tempo avere un \keyword{miglioramento permanente e continuo}, degli stessi, consiste nell'adottare un \english{\glossarySingolarTerm{framework}} quale, per esempio, \ac{Information-Technology-Infrastructure-Library} per la gestione del proprio dipartimento \acs{Information-Technology}.

Lo scopo che questo documento si prefigge di raggiungere consiste nella progettazione e la conseguente implementazione di una particolare funzionalità presente nel \english{framework}, la funzionalità di \keyword{\english{Service Desk}}. Verranno inoltre illustrati alcuni dei processi, che direttamente, ha lo scopo di amministrare.

Essi sono: 

\begin{itemize}
\item{\acf{Event-Management}}
\item{\acf{Incident-Management}}
\item{\acf{Request-Fulfillment}}
\end{itemize}

%------------------------------------------------
%
% Structure.tex
%
% This file contains the structure of the
% designed Service Desk
%------------------------------------------------
\section[Struttura del Service Desk]{struttura del service desk}
\label{sd-structure}
Vi sono modi differenti per strutturare e collocare la funzionalità di \english{Service Desk} e la soluzione corretta varia in base alle esigenze dell'organizzazione che intende dotarsi di tale funzionalità. I possibili tipi di \english{Service Desk} sono:

\begin{itemize}
\item{locale}
\item{centralizzato}
\item{virtuale}
\end{itemize}

In seguito vengono descritti brevemente le tre tipologie al fine di far comprendere quali sono le motivazioni che hanno portato alla scelta di una specifica tipologia.

\subsection[Service Desk Locale]{service desk locale}
\label{sd-local-sd}
Un \english{Service Desk} si definisce locale se si trova all'interno o fisicamente vicino alla comunità di utenti a cui deve prestare servizio. Questo spesso aiuta nella comunicazione e dà una presenza ben visibile, che può essere apprezzata dagli utenti, ma spesso risulta essere inefficiente e costoso nell'uso delle risorse. Come ad esempio personale che rimane in attesa di richieste ma il volume di esse non ne giustifica una presenza fissa.

Ci possono, tuttavia, essere alcune valide ragioni per il mantenimento di questa tipologia anche se i volumi delle richieste non lo giustificherebbero. Le ragioni possono includere:

\begin{itemize}
\item{differenze di lingua, cultura o politica;}
\item{differenze di fuso orario;}
\item{gruppi specializzati di utenti;}
\item{l'esistenza di servizi personalizzati/specializzati che richiedono un conoscenza specifica;}
\item{la presenza di \acs{Very-Important-Person}/utenti critici.}
\end{itemize}

\subsection[Service Desk Centralizzato]{service desk centralizzato}
\label{sd-centralized-sd}
E' possibile ridurre il numero di \english{Service Desk} fondendoli in una singola locazione (oppure in un numero piccolo di locazioni) collocando lo staff in uno o più \english{Service Desk} centralizzati.

Questo può essere \keyword{molto più efficiente} ed \keyword{economico}, consentendo un piccolo aumento di personale se vi è un numero elevato di richieste in ingresso, e può anche portare a livelli di competenza maggiormente elevati attraverso una maggior familiarizzazione con i servizi erogati.

Potrebbe comunque essere necessario mantenere una qualche presenza locale per gestire i requisiti di supporto fisico, ma tale personale può essere gestito direttamente dall'unità centrale.

\subsection[Service Desk Virtuale]{service desk virtuale}
\label{sd-virtual-sd}
Attraverso l'uso della tecnologia, in particolare internet, a l'uso di strumenti appropriati è possibile fornire l'impressione di un singolo, centralizzato \english{Service Desk}, quando in realtà il personale può essere sparso o localizzato in numerose aree geografiche.

Questo consente la possibilità di lavorare da casa oppure gruppi di sostegno secondari, decentralizzati o esternalizzati. E' comunque doveroso osservare che sono necessarie salvaguardie in tutte queste circostanze al fine di assicurare la \keyword{coerenza}, l'\keyword{uniformità} e la \keyword{qualità} del servizio erogato.

\subsection[Scelta del proponente]{scelta del proponente}
Dopo aver valutato attentamente le tre tipologie possibili di \english{Service Desk} e la collocazione geografica dei presidi dell'istituto Gaetano Pini, il proponente ritiene sia più corretta l'implementazione di un \english{Service Desk} di tipo ibrido \keyword{locale-centralizzato}.

I presidi posseduti dall'istituto si trovano rispettivamente:

\begin{itemize}
\item{piazza Cardinal Ferrari 1 Milano}
\item{via Isocrate Milano}
\end{itemize}

Data la vicinanza dei due presidi non si ritiene necessario duplicare le risorse di \english{Service Desk} per ogni presidio, ma si istituirà la sede principale della funzionalità presso uno degli uffici nel plesso principale (piazza Cardinal Ferrari 1) e spostare un sottoinsieme dello staff tecnico, quello/i che si occupano del/i servizio/i che risiedono nel secondo plesso (via Isocrate).

Otteniamo cosi un unico \english{Service Desk} logico in cui sarà più facile la gestione delle risorse con il vantaggio che i gruppi specializzati risiedano vicino alle aree in cui i servizi sono maggiormente utilizzati, diminuendo cosi eventuali tempi per interventi in loco.

%------------------------------------------------
%
% Recipient.tex 
%
% This section show the recipient of this
% document.
%------------------------------------------------
\section[Diretti interessati]{diretti interessati}
\label{abs-recipient}
Il presente documento è rivolto ai membri del consiglio di amministrazione dell'Ospedale Gaetano Pini di Milano. Alla C.A. dei dottori:

\begin{itemize}
\item{Dott. Amedeo Tropiano -- Direttore Generale}
\item{Dott.ssa Loredana Maspes -- Direttore Amministrativo}
\item{Dott. Nunzio Angelo Buccino -- Direttore Sanitario}
\end{itemize}

%------------------------------------------------
%
% Hightlighting.tex 
%
% This section describe the hightlightining
% used in the document.
%------------------------------------------------
\section[Convenzioni tipografiche]{convenzioni tipografiche}
\label{abs-hightlighting}
Nel seguente documento si sono introdotte le seguenti convenzioni tipografiche al fine di consentire una lettura scorrevole dello stesso.

Esse sono:

\begin{itemize}
\item{\textbf{grassetto} per evidenziare i termini più importanti presenti nel paragrafo;}
\item{\textit{corsivo} per evidenziare i termini in lingua inglese;}
\item{\underline{sottolineato} per la prima occorrenza di termini presenti nel glossario finale.}
\end{itemize}