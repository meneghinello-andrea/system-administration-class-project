%------------------------------------------------
%
% Times_Contact.tex
%
% This file contains times when the Service desk
% is operative and the contact mode
%------------------------------------------------
\section[Operatività e modalità di contatto]{operatività e modalità di contatto}
\label{sd-operativity}
In questa sezione viene illustrato l'orario di operatività e le modalità di contatto assunte dalla funzionalità di \english{Service Desk} che si vuole implementare.

\subsection[Tempi di operatività]{tempi di operatività}
\label{sd-operativity-time}
Dopo aver appreso i tempi di operatività sanitaria ed i livelli di criticità dei sistemi adottati dall'\entity{} l'ente proponente ha studiato gli orari di operatività, riportati in Tabella \ref{sd-operativity-working-table} per le giornate feriali.

\begin{center}
\begin{longtable}{| p{4cm} | p{8cm} |}
\caption[Orari di operatività feriale]{Orari di lavoro giorni feriali}
\label{sd-operativity-working-table}\\
\hline
\multicolumn{1}{| c |}{\textbf{Giorno}} & \multicolumn{1}{| c |}{\textbf{Orario}}\\
\endfirsthead
\hline
\multicolumn{1}{| c |}{\textbf{Giorno}} & \multicolumn{1}{| c |}{\textbf{Orario}}\\
\endhead
\hline
Lunedì & \multicolumn{1}{| c |}{7.30 - 18.00}\\
\hline
Martedì & \multicolumn{1}{| c |}{7.30 - 18.00}\\
\hline
Mercoledì & \multicolumn{1}{| c |}{7.30 - 18.00}\\
\hline
Giovedì & \multicolumn{1}{| c |}{7.30 - 18.00}\\
\hline
Venerdì & \multicolumn{1}{| c |}{7.30 - 18.00}\\
\hline
Sabato & \multicolumn{1}{| c |}{7.30 - 18.00}\\
\hline
Domenica & \multicolumn{1}{| c |}{7.30 - 18.00}\\
\hline
\end{longtable}
\end{center}

Per quanto concerne i giorni festivi la funzionalità di \english{Service Desk} osserverà un orario di operatività ridotto, riportato in Tabella \ref{sd-operativity-holiday-table}, in quanto si è osservato che il numero di chiamate entranti in giorni festivi è ridotto rispetto al normale, dovuto al fatto che il numero di visite/ricoveri è minore.

\begin{center}
\begin{longtable}{| p{4cm} | p{8cm} |}
\caption[Orari di operatività festivo]{Orari di lavoro giorni festivi}
\label{sd-operativity-holiday-table}\\
\hline
\multicolumn{1}{| c |}{\textbf{Giorno}} & \multicolumn{1}{| c |}{\textbf{Orario}}\\
\endfirsthead
\hline
\multicolumn{1}{| c |}{\textbf{Giorno}} & \multicolumn{1}{| c |}{\textbf{Orario}}\\
\endhead
\hline
Lunedì & \multicolumn{1}{| c |}{9.00 - 12.00 / 14.00 - 16.00}\\
\hline
Martedì & \multicolumn{1}{| c |}{9.00 - 12.00 / 14.00 - 16.00}\\
\hline
Mercoledì & \multicolumn{1}{| c |}{9.00 - 12.00 / 14.00 - 16.00}\\
\hline
Giovedì & \multicolumn{1}{| c |}{9.00 - 12.00 / 14.00 - 16.00}\\
\hline
Venerdì & \multicolumn{1}{| c |}{9.00 - 12.00 / 14.00 - 16.00}\\
\hline
Sabato & \multicolumn{1}{| c |}{9.00 - 12.00 / 14.00 - 16.00}\\
\hline
Domenica & \multicolumn{1}{| c |}{9.00 - 12.00 / 14.00 - 16.00}\\
\hline
\end{longtable}
\end{center}

Si ricorda che è compito poi dei \english{Service Desk Supervisor} suddividere in turni l'orario di operatività della funzione di \english{Service Desk} al fine di coprire le fasce orarie di attività sulla base di esigenze specifiche.

Si ricorda che il \english{Service Desk Manager} ed i \english{Service Desk Supervisor} dovranno inoltre garantire la propria reperibilità a fronte di gravi incidenti anche all'esterno degli orari di operatività del \english{Service Desk}.

Si ricorda inoltre che tutte le richieste che giungeranno oltre il termine della giornata di lavoro, sia festiva che feriale, saranno prese in carico il primo giorno successivo secondo gli orari elencati nelle Tabelle \ref{sd-operativity-working-table} e \ref{sd-operativity-holiday-table}.

\subsection[Metodi di contatto]{metodi di contatto}
\label{sd-contact-mode}
La funzione di \english{Service Desk} sarà contattabile attraverso diverse modalità per andare in contro alle esigenze delle diverse tipologie di utenti.

Le modalità previste sono:

\begin{itemize}
\item{portale web}
\item{contatto telefonico}
\item{contatto tramite e-mail}
\end{itemize}

\subsubsection[Portale Web]{portale web}
Viene predisposto un portale web, accessibile solamente attraverso la \english{intranet} locale, in cui gli utenti potranno contattare la funzione di \english{Service Desk} per:

\begin{itemize}
\item{effettuare richieste;}
\item{verificare la soluzione a errori noti;}
\item{visionare statistiche sulla funzione di \english{Service Desk}.}
\end{itemize}

Il portale sarà reperibile, anche all'esterno dei tempi di operatività (vedi Sezione \ref{sd-operativity-time}), al seguente indirizzo: 

\begin{center}
http://gpini.com/service-desk
\end{center}

\subsubsection[Contatto telefonico]{contatto telefonico}
E' inoltre possibile contattare telefonicamente la funzione di \english{Service Desk} al fine di parlare con un operatore. 

Il contatto telefonico prevede da parte dell'utente una scelta con quale tipo di operatore avere il colloquio sulla base dell'esigenza (vedi Sezione \ref{sd-resources-categories}). L'utente sarà aiutato in questa fase da una voce registrata che lo guiderà.

La funzione di \english{Service Desk} è reperibile al seguente numero interno: +39 555 555501.

\subsubsection[Contatto tramite e-mail]{contatto tramite e-mail}
La funzione di \english{Service Desk} espone per gli utenti anche un indirizzo e-mail presso cui inoltrare le proprie richieste.

L'indirizzo e-mail è il seguente:

\begin{center}
service.desk@gpini.com
\end{center}

Si vuole far notare che è compito dello strumento di supporto (vedi Sezione REF) al \english{Service Desk} convertire e redirigere tali e-mails presso la schermata di lavoro dello staff tecnico.