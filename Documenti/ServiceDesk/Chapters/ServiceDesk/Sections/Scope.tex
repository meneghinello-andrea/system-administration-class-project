%------------------------------------------------
%
% Scope.tex 
%
% This section illustrates the scope of a service
% desk.
%------------------------------------------------
\section[Scopo del Service Desk]{scopo del service desk}
\label{sd-scope}
Lo scopo principale di un \english{Service Desk} è quello di essere l'unico \ac{Service-Point-Of-Contact} tra gli utenti, che hanno necessità di assistenza/aiuto nell'utilizzo dei servizi \acs{Information-Technology}, e lo staff tecnico che risponde alle loro richieste.

Oltre ad essere una chiaro abilitatore di \english{business} (\keyword{\english{business enabler}}) esso attiva una crescita rigorosa dell'organizzazione che lo contiene. Inoltre le prestazioni di un \english{Service Desk} forniscono un'indicazione sul \keyword{livello generale di ``salute''} del dipartimento \acs{Information-Technology}.

\subsection[Benefici]{benefici}
\label{sd-benefits}
La funzione di \english{Service Desk} fornisce all'intera organizzazione i seguenti benefici:

\begin{itemize}
\item{incremento della percezione e soddisfazione del servizio clienti;}
\item{aumento dell'accessibilità ad assistenza/aiuto, della comunicazione e dell'informazione;}
\item{miglior qualità ed una veloce risposta delle richieste dei clienti/utenti;}
\item{incentiva il lavoro di gruppo e la comunicazione;}
\item{maggior attenzione ed un approccio proattivo alla fornitura di servizi;}
\item{miglior gestione e controllo dell'intera infrastruttura \acs{Information-Technology};}
\item{incremento nell'uso delle risorse di supporto \acs{Information-Technology}.}
\end{itemize}

\subsubsection[Incremento della percezione e soddisfazione dell'utente]{incremento della percezione e soddisfazione dell'utente}
Gli utenti dei servizi \acs{Information-Technology} eseguiranno le loro mansioni in modo molto più produttivo sapendo che per qualsiasi bisogno/esigenza, possono contare su uno staff di esperti che li aiuti ad uscire dalle difficoltà che potrebbero incontrare.

\subsubsection[Aumento dell'accessibilità, della comunicazione e dell'informazione]{aumento dell'accessibilità, della comunicazione e dell'informazione}
Essendo, per definizione, la funzione di \english{Service Desk} l'unico \ac{Service-Point-Of-Contact}, tra utenti dei servizi \acs{Information-Technology} e lo staff tecnico del dipartimento, l'intera organizzazione trae un enorme beneficio in quanto in qualsiasi caso di necessità si rivolgeranno a quest'unico punto.

Si eviterà cosi di scaricare sugli utenti l'onere della scelta di quale specifica funzione del dipartimento \acs{Information-Technology} interrogare in base al loro tipo di richiesta.

Quest'onere sarà invece scaricato sullo staff tecnico del \english{Service Desk}, in quanto possiede una maggiore visione dell'intero sottosistema \acs{Information-Technology} presente nell'istituto Gaetano Pini e potrà quindi redirigere la richiesta nel modo più opportuno.

\subsubsection[Miglior qualità nelle risposte]{miglior qualità nelle risposte}
Essendo le richieste effettuate ad uno staff tecnico che conosce, sempre meglio, l'ambiente \acs{Information-Technology} in cui sono ospitati i servizi, esso sarà in grado di fornire delle risposte, sia in merito ad incidenti che semplici richieste di servizio, agli utenti che siano ad alto contenuto qualitativo, generando cosi un elevato grado di soddisfazione da parte di questi ultimi.

\subsubsection[Incentiva il lavoro di gruppo e la comunicazione]{incentiva il lavoro di gruppo e la comunicazione}
La funzione di \english{Service Desk} incentiva il lavoro di gruppo al fine di migliorare l'intervento sulle richieste effettuate dagli utenti e con ciò si ha come conseguenza diretta un aumento della comunicazione all'interno dello staff tecnico e di conseguenza anche con gli utenti.

Il vantaggio generale che se ne trae è che la conoscenza non viene confinata in una sotto area del dipartimento \acs{Information-Technology} ma questa si espande a tutti gli utenti che ne sono interessati.

\subsubsection[Maggior attenzione ed approccio proattivo nella fonrnitura dei servizi]{maggior attenzione ed approccio proattivo nella fornitura dei servizi}
Il \english{Service Desk} non ha solamente lo scopo di rispondere ad incidenti che possano verificarsi all'interno dell'ambiente \acs{Information-Technology}, ma esso effettua anche monitoraggio sullo stesso consentendo quindi di recuperare situazioni prima ancora che queste possano generare incidenti percepiti dagli utenti.

Questo consente agli utenti di percepire un maggior livello di servizio in quanto vedranno servizi erogati molto stabili e duraturi, che consentiranno loro di svolgere le loro mansioni nel modo più fluido possibile.

\subsubsection[Miglior gestione e controllo dell'infrastruttura]{miglior gestione e controllo dell'infrastruttura}
Attraverso l'uso di strumenti adeguati il personale del \english{Service Desk} potrà gestire in modo molto fluido i propri compiti che variano dall'intervento a seguito di un \glossarySingolarTerm{incidente} segnalato, al monitoraggio dell'ambiente \acs{Information-Technology} fino alla gestione delle richieste di servizio.

Tali strumenti consentiranno di mantenere ogni intervento \keyword{tracciato} e \keyword{correttamente archiviato} al fine di velocizzare l'accesso in caso di future consultazioni.

\subsubsection[Incremento dell'uso delle risorse di supporto]{Incremento dell'uso delle risorse di supporto}
Monitorando l'intero ambiente \acs{Information-Technology} la funzione di \english{Service Desk} può osservare l'uso complessivo delle risorse \acs{Information-Technology} ed attivare gli opportuni processi, quali \ac{Availability-Management} e \ac{Capacity-Management}, qualora le risorse offerte non siano sufficienti oppure, al contrario, troppo elevate con conseguente spreco.

\subsection[Assicurare i risultati]{assicurare i risultati}
\label{sd-ensuring-results}
Nella realtà economica odierna spesso la riduzione dei costi è una necessità, ed i gruppi di supporto agli utenti sono generalmente i primi a subirli. E' perciò necessario assicurare che i servizi da loro offerti siano \keyword{chiaramente definiti} e \keyword{allineati} con i bisogni che il \english{business} richiede.

Al fine di introdurre e successivamente mantenere una funzionalità di \english{Service Desk} di successo, è essenziale che:

\begin{itemize}
\item{le necessità del \english{business} siano correttamente comprese;}
\item{le richieste degli utenti siano chiaramente comprese;}
\item{siano effettuati investimenti nella formazione dello staff tecnico;}
\item{obiettivi e risultati sino chiaramente definiti;}
\item{i livelli di servizio siano praticabili, accettati e regolarmente rivisiti.}
\end{itemize}

\subsection[Obiettivi]{obiettivi}
\label{sd-objectives}
L'obiettivo primario di un \english{Service Desk} consiste nel \keyword{ripristinare i normali livelli di servizio agli utenti il più velocemente possibile}. 

In questo contesto il ``ripristino dei normali livelli di servizio'' è inteso nel modo più ampio possibile. Potrebbe comportare il ripristino a seguito di un guasto tecnico, oppure si potrebbe trattare di soddisfare una richiesta di servizio o ancora più semplicemente rispondere ad una domanda. 

Deve essere fatto tutto ciò che risulti essere necessario al fine di consentire agli utenti di ritornare alle proprie mansioni in modo soddisfacente.

Responsabilità specifiche comprenderanno:

\begin{itemize}
\item{il tracciamento tutti i dettagli riguardanti le richieste entranti;}
\item{la fornitura di una prima linea di supporto;}
\item{la risoluzione degli incidenti ``semplici'';}
\item{la ``chiusura'' di tutte le richieste che risultino essere soddisfatte;}
\item{la conduzione di sondaggi sulla soddisfazione percepita dagli utenti;}
\item{la comunicazione con gli utenti;}
\item{l'aggiornamento del \ac{Configuration-Management-DataBase}.}
\end{itemize}

\subsubsection[Tracciamento dettagli delle richieste]{tracciamento dettagli delle richieste}
Il poter tracciare correttamente tutti i dettagli rilevanti delle richieste (incidenti/richieste di servizio) che giungono dagli utenti al \english{Service Desk}, consente di poter fornire molto più velocemente, la prima linea di supporto prevista all'interno del \english{Service Desk}.

Il risultato ottenuto dal tracciamento delle richieste è una base di dati di enorme valore per il \english{Service Desk}, chiamata in gergo tecnico \ac{Knowledge-Base}, in quanto consentirà di fornire una prima linea di supporto \keyword{più reattiva} a medio lungo termine. Questo perché con il passare del tempo è probabile che alcune richieste possano avere delle similitudini con altre già risolte e consultando questo \english{database} potrebbe non servire la fase l'analisi.

Le richieste entranti, di qualunque tipo esse siano, saranno \keyword{categorizzate} e verrà loro fornita un \keyword{priorità} affinché possano essere gestite dal personale più opportuno e nei tempi migliori, dato che le richieste saranno di natura differente ed alcune più urgenti di altre.

\subsubsection[Fornitura della prima linea di supporto]{fornitura della prima linea di supporto}
Generalmente la prima linea di supporto, presente in un \english{Service Desk}, ha lo scopo di investigare sulle possibili/probabili cause che hanno portato al verificarsi dell'incidente, e scalare ad uno staff più avanzato quando essi non riescano, in breve tempo, a ripristinare il servizio. 

Le informazioni che comunque hanno ricavato saranno successivamente allegate alla richiesta entrante, e potranno fornire una base di partenza per gruppi di analisi/diagnosi del problema più avanzati.

\subsubsection[Risoluzione degli incidenti semplici]{risoluzione degli incidenti semplici}
Vengono considerati come ``semplici'' quegli incidenti che non necessitano di un supporto più avanzato per la loro risoluzione. Tali incidenti presentano, come proprietà, ``sintomi'' uguali oppure molto simili ad incidenti avvenuti nel passato.

Per la loro risoluzione la prima linea di supporto non dovrà fare altro che leggere le informazioni di risoluzione collegate a quegli incidenti, simili, che ora risultano essere chiusi.

Tali informazioni sono sempre reperibili all'interno del \ac{Known-Error-DataBase} dove risiedono gli \glossaryPluralTerm{errore}.

\subsubsection[Chiusura delle richieste soddisfatte]{chiusura delle richieste soddisfatte}
E' compito del membro che si è assunto la responsabilità di gestione della richiesta di effettuare correttamente la procedura di chiusura quando essa risulterà essere soddisfatta.

La richiesta potrà veramente definirsi terminata, quindi chiusa, solamente quando una soluzione al problema viene trovata e questa soluzione \keyword{è confermata} come risolutiva da parte dell'utente che ha aperto la richiesta presso il \english{Service Desk}.

\subsubsection[Conduzione di sondaggi sulla soddisfazione percepita]{conduzione di sondaggi sulla soddisfazione percepita}
Un altro compito che lo staff del \english{Service Desk} deve svolgere è quello di preparare e condurre dei sondaggi periodici che hanno lo scopo di determinare il livello di servizio percepito dagli utenti.

Dato che tutto il \english{framework} \ac{Information-Technology-Infrastructure-Library} ruota attorno al principio cardine del \glossarySingolarTerm{deming} con questa fase si vuole controllare che il lavoro svolto dallo staff del \english{Service Desk} su un predeterminato periodo temporale sia in linea con i requisiti al fine di scovare anomalie di funzione/processo per porvi rimedio.

\subsubsection[Comunicazione con gli utenti]{comunicazione con gli utenti}
Dopo che la richiesta è giunta al \english{Service Desk} e che un membro dello staff tecnico ne ha assunto la responsabilità è necessario che esso si assuma anche il compito di \keyword{mantenere costantemente aggiornato} l'utente della richiesta in merito agli avanzamenti di stato.

Questa procedura è resa molto più semplice attraverso l'uso di strumenti \english{software} specializzati che consentono di eseguire questa comunicazione in automatico quando la richiesta di assistenza/aiuto cambia stato o viene aggiornata.

\subsubsection[Aggiornamento del CMDB]{aggiornamento del CMDB}
Sotto la direzione e l'approvazione del responsabile del processo di \ac{Configuration-Management} lo staff tecnico ha il compito di aggiornare il \ac{Configuration-Management-DataBase} ossia il \english{database} contenente la descrizione della struttura attuale e passata dei \glossarySingolarTerm{configuration-item} presenti.

L'aggiornamento di questa base di dati potrebbe risultare necessaria per i seguenti motivi:

\begin{itemize}
\item{potrebbe essere stato sviluppato un \glossarySingolarTerm{workaround} per consentire all'utente di poter utilizzare il servizio finché il processo di \ac{Problem-Management} non trova una ``cura'' per le cause che hanno scatenato l'incidente.}
\item{per risolvere il problema è stato necessario modificare un \ac{Configuration-Item} quindi, esso non è più configurato come prima che l'incidente avvenisse.}
\end{itemize}