%------------------------------------------------
%
% GlossaryDatabase.tex
%
% This file contains the terms that are present
% in the glossary of the document.
%------------------------------------------------
\newglossaryentry{efficace}
{
	name		= {efficace},
	plural		= {efficaci},
	description = {Un servizio che produce pienamente l'effetto richiesto o desiderato}
}

\newglossaryentry{efficiente}
{
	name		= {efficiente},
	plural		= {efficienti},
	description = {Raggiungimento dell'obiettivo preposto nel modo migliore possibile ossia usando al meglio le risorse a propria disposizione}
}

\newglossaryentry{framework}
{
	name		= {framework},
	plural		= {frameworks},
	description = {In informatica, specialmente nello sviluppo \english{software}, è un'architettura logica di supporto su cui un \english{software}/servizio può essere progettato e realizzato spesso facilitandone lo sviluppo}
}

\newglossaryentry{incidente}
{
	name		= {incidente},
	plural		= {incidenti},
	description	= {Un incidente (\english{incident}) è un qualsiasi evento che non fa parte dell'operatività \english{standard} del servizio e che causa una riduzione o un'interruzione del servizio offerto}
}

\newglossaryentry{overflow}
{
	name		= {overflow},
	plural		= {overflow},
	description = {Condizione che accade quando un calcolo produce un risultato che è più grande di quello che può essere memorizzato}
}

\newglossaryentry{release}
{
	name		= {release},
	plural		= {releases},
	description = {Nell'ambito dello sviluppo di un servizio una \english{release} è una specifica versione dello stesso resa disponibile agli utenti finali. E' univocamente identificata in modo che sia possibile distinguerla dalle precedenti. Convenzionalmente si suddividono in \english{release} maggiori (\english{Major Release}) quando le differenze dalla precedente versione riguardano sostanziali evoluzioni delle funzionalità offerte e \english{release} minori (\english{Minor Release}) quando le differenze riguardano principalmente correzioni di malfunzionamenti}
}

\newglossaryentry{rollout}
{
	name		= {roll out},
	plural		= {roll out},
	description = {Nel campo dei sistemi informativi il termine \english{roll out} è comunemente impiegato per identificare il processo con cui un servizio viene attivato a partire da una istanza iniziale. Durante il \english{roll out} del servizio si trovano le fasi progettuali di \english{training}, di collaudo e di messa in esercizio}
}

\newglossaryentry{workaround}
{
	name		= {workaround},
	plural		= {workarounds},
	description	= {Si tratta di una correzione temporanea ad un incidente oppure una sequenza di azioni alternative a quella che produce l'incidente che però rende utilizzabile, anche se in versione ridotta, il servizio}
}