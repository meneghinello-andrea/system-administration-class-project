%------------------------------------------------
%
% Match.tex 
%
% This section illustrates the difference
% between a service desk and an help desk
% and illustrate why a service desk is
% better
%------------------------------------------------
\section[Service Desk vs Help Desk]{service desk vs help desk}
\label{sd-sd-vs-hd}
L'ente offerente intende proporre all'amministrazione dell'istituto Gaetano Pini l'implementazione della più moderna funzionalità di \english{Service Desk}, inclusa nel \english{framework} \ac{Information-Technology-Infrastructure-Library} v.3, anziché la funzionalità di \english{Help Desk} presente nella precedente versione del \english{framework}.

Viene ora fornita una breve panoramica sulla funzione di \english{Help Desk} per focalizzarsi poi sui vantaggi offerti dalla più moderna piattaforma.

\subsection[Funzione di Help Desk]{funzione di help desk}
\label{sd-hd}
Nella precedente versione del \english{framework} \ac{Information-Technology-Infrastructure-Library}, la versione 2, il \english{software} è il punto focale attorno a cui ruotano tutti i processi.

La funzionalità di \english{Help Desk} si focalizza quindi sulle seguenti aree:
\begin{itemize}
\item{gestione di incidenti \english{software};}
\item{collaborazione con il processo di \ac{Change-Management}.}
\end{itemize}

Oggigiorno però le istituzioni chiedono molto di più, esse richiedono servizi. Questa versione del \english{framework} non riesce però a soddisfare questa richiesta.

\subsection[Funzione di Service Desk]{funzione di service desk}
\label{sd-sd}
Con la più recente versione del \english{framework}, la versione 3, il punto focale è diventato il \keyword{servizio}, riuscendo cosi a rispondere alle nuove esigenze poste dalle istituzioni.

Data la chiara intenzione, esposta nel capitolato d'appalto, di rinnovare ed allinearsi a quanto di più moderno esista per offrire livelli di servizio che risultino essere il più \keyword{\glossaryPluralTerm{efficace}} ed \keyword{\glossaryPluralTerm{efficiente}} possibile, l'ente proponente nel resto della proposta illustrerà una possibile implementazione della funzione di \english{Service Desk} in quanto, a nostro avviso risponde pienamente alle richieste.

Nel seguito del capitolo l'ente proponente illustra quali sono gli scopi che la funzionalità possiede, specificando benefici, risultati ed obiettivi.

Vengono inoltre illustrati i ruoli e le responsabilità che dovranno essere assunti dai membri dello staff che apparterrà a questa divisione del dipartimento \acs{Information-Technology}.

Successivamente è illustrata la struttura logica che dovrà essere implementata per garantire agli utenti quanto specificato nella proposta.

Infine, e non meno importante, verrà fornita una breve panoramica sugli strumenti di cui dovrà dotarsi la funzione di \english{Service Desk}.