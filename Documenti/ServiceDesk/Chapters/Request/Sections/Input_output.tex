%------------------------------------------------
%
% Input_output.tex 
%
% This section introduces the input and output
% of the request fulfillment process.
%------------------------------------------------
\section[Input e output del processo]{input e output del processo}
\label{rf-io}
In questa sezione vengono illustrati gli \english{input} e gli \english{output} del processo, specificando da chi sono ricevuti e a chi sono rivolti.

\subsection[Input del processo]{input del processo}
\label{rf-io-input}
In Tabella \ref{rf-io-input-table} sono elencati gli \english{input} del processo e da chi sono ricevuti.

\begin{center}
\begin{longtable}{| p{6cm} | p{7cm} |}
\caption{Input del processo}
\label{rf-io-input-table}\\
\hline
\multicolumn{1}{| c |}{\textbf{Input}} & \multicolumn{1}{| c |}{\textbf{Da}}\\
\hline
\endfirsthead
\hline
\multicolumn{1}{| c |}{\textbf{Input}} & \multicolumn{1}{| c |}{\textbf{Da}}\\
\hline
\endhead
Richiesta di servizio (scritta) & Utenti\\
\hline
\end{longtable}
\end{center}

Gli \english{input} che esso riceve avvengono tramite \english{ticket}, essi sono ricevuti dal \english{Service Desk} e successivamente dirottati a questo processo.

I campi e gli attributi che lo compongono sono illustrati nelle Tabelle \ref{im-io-input-ticket-common-table}, \ref{im-io-input-ticket-upgrade-table}, \ref{im-io-input-ticket-attachment-table}, \ref{im-io-input-ticket-resolution-table} e \ref{im-io-input-ticket-history-table}.

\subsection[Output del processo]{output del processo}
\label{rf-io-output}
In Tabella \ref{rf-io-output-table} sono elencati gli \english{output} del processo e a chi sono rivolti.

\begin{center}
\begin{longtable}{| p{6cm} | p{7cm} |}
\caption{output del processo}
\label{rf-io-output-table}\\
\hline
\multicolumn{1}{| c |}{\textbf{Input}} & \multicolumn{1}{| c |}{\textbf{Da}}\\
\hline
\endfirsthead
\hline
\multicolumn{1}{| c |}{\textbf{Input}} & \multicolumn{1}{| c |}{\textbf{Da}}\\
\hline
\endhead
Notifiche agli utenti quando la richiesta è conclusa. & Utenti\\
\hline
La soluzione incontra le richieste dell'utente. & Utenti\\
\hline
\end{longtable}
\end{center}