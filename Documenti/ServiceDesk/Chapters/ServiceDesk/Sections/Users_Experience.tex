%------------------------------------------------
%
% Users_Experience.tex 
%
% This section illustrates the user experience
% that the service desk qant to reach.
%------------------------------------------------
\section[Esperienza utente]{esperienza utente}
\label{sd-users-experience}
L'obiettivo di questa parte del documento è quella di fornire alcune semplici linee guida (\english{best practies}) che mirano a rendere il \english{Service Desk} un'entità vicina alle esigenze degli utenti e di facile utilizzo, in quanto si vuole implementare una funzionalità che sia gradevolmente interrogata a bisogno dagli utenti.

Migliorare l'esperienza dell'utente finale, non si limita a beneficio dell'utente ma aiuta nell'erogazione dei servizi e rafforza molto di più questa funzionalità.

Alcuni suggerimenti per migliorare l'esperienza utente sono:

\begin{itemize}
\item{fornire il materiale necessario allo staff tecnico;}
\item{rendere le informazioni necessarie sempre disponibili;}
\item{offrire agli utenti differenti canali di comunicazione;}
\item{offrire allo staff tecnico \english{software} di tipo \acs{Software-as-a-Service};}
\item{utilizzare la funzionalità di \english{desktop} remoto per risoluzioni più veloci;}
\item{aiutare gli utenti ad aiutarsi.}
\end{itemize}

\subsection[Fornire il materiale necessario]{fornire il materiale necessario}
\label{sd-users-experience-material}
Fornendo allo staff tecnico del \english{Service Desk} il materiale di cui necessita per operare, esso aumenterà di conseguenza la propria produttività. Come risultato si ottiene una funzionalità che è molto veloce nel fornire aiuto, apparendo cosi versatile agli occhi degli utenti.

Un \english{Service Desk} funzionale ha bisogno di:

\begin{itemize}
\item{strumenti per la gestione delle richieste;}
\item{strumenti per la gestione delle priorità;}
\item{strumenti per la gestione dei \english{report};}
\item{strumenti per mantenere i registri.}
\end{itemize}

Nella sezione \ref{sd-tools} si propone l'adozione di uno strumento \english{software} che \english{agevola} e \english{velocizza} le fasi di lavoro all'interno del \english{Service Desk}.

\subsection[Accessibilità delle informazioni]{accessibilità delle informazioni}
\label{sd-users-experience-accessibility}
Lo staff tecnico del \english{Service Desk} deve avere facile accesso alle informazioni sui problemi noti, le relative soluzioni, le recenti modifiche all'infrastruttura \acs{Information-Technology} e quali strumenti/servizi sono assegnati ad un particolare utente.

Attraverso il facile accesso alle informazioni precedenti qualunque membro dello staff saprà come gestire al meglio una richiesta.

Questo comporta un aumento della velocità di risposta del \english{Service Desk} con conseguente miglioramento della percezione da parte dell'utilizzatore finale.

\subsection[Canali di comunicazione]{canali di comunicazione}
\label{sd-users-experience-communication}
La fornitura di differenti canali di comunicazione (vedi sezione \ref{sd-contact-mode}) agli utenti finali rende la funzione di \english{Service Desk} maggiormente accessibile a diverse categorie di utenti.

Vista però dal punto di vista del \english{Service Desk} questo potrebbe generare confusione e rallentare il lavoro.

Tuttavia l'uso di strumenti appropriati e automatici è possibile far si che le richieste entranti da differenti canali di comunicazione siano automaticamente reindirizzate in un unico canale visibile al personale del \english{Service Desk}.

Il \english{software} proposto (vedi Sezione \ref{sd-tools}) fornisce una funzionalità che converte automaticamente le \english{e-mails} ricevute su uno specifico indirizzo in richieste di assistenza/aiuto (\english{ticket}).

\subsection[Fornitura di software SaaS]{fornitura di software SaaS}
\label{sd-users-experience-saas}
Utilizzando come \english{software} di supporto, alle proprie mansioni, un \ac{Software-as-a-Service} (vedi Sezione REF) l'intero dipartimento \acs{Information-Technology} non deve preoccuparsi quando si verificano manutenzioni \english{hardware}, installazioni di \english{\glossarySingolarTerm{patch}}, \english{fix} di sicurezza e aggiornamenti \english{software} in quanto sono a carico del fornitore del servizio.

Il fornitore del servizio \english{software} assicura che ogni membro dello staff tecnico lavora con la versione corretta dello stesso e che aggiornamenti, \english{pathces}, e risoluzione di problemi siano installati automaticamente.

\subsection[Utilizzo della funzionalità di desktop remoto]{utilizzo della funzionalità di desktop remoto}
\label{sd-users-experience-remote-desktop}
Alcuni incidenti possono essere risolti attraverso l'uso della funzionalità di \english{Desktop} remoto con gli utenti, cosi che i membri dello staff tecnico non debbano raggiungere, ogni volta, la postazione di lavoro dell'utente.

Il tempo risparmiato può essere significativo quando lo staff tecnico riesce a risolvere il problema dalla propria postazione di lavoro, e l'utente può tornare ad essere produttivo con il più piccolo \english{downtime} possibile.

\subsection[Aiutare gli utenti ad aiutarsi]{aiutare gli utenti ad aiutarsi}
\label{sd-user-experience-help}
Rendendo autonomi gli utenti nella risoluzione di piccole difficoltà, che presentano una soluzione nota, fanno risparmiare tempo allo staff del \english{Service Desk} che può quindi dedicarsi alla risoluzione di incedenti più gravi.

Il \english{software} proposto (vedi Sezione \ref{sd-tools-knowledge-base}) prevede una sezione, interrogabile dagli utenti, in cui possono trovare spiegazioni ai problemi comuni.