%------------------------------------------------
%
% Match.tex 
%
% This section illustrates the difference
% between a service desk and an help desk
% and illustrate why a service desk is
% better
%------------------------------------------------
\section[Service Desk vs Help Desk]{service desk vs help desk}
\label{sd-sd-vs-hd}
L'ente offerente intende proporre all'amministrazione dell'ospedale Gaetano Pini l'implementazione della più moderna funzionalità di \english{Service Desk}, inclusa nel \english{framework} \ac{Information-Technology-Infrastructure-Library} v.3, anziché la funzionalità di \english{Help Desk} presente nella precedente versione del \english{framework}.

Viene ora fornita una breve panoramica sulla funzione di \english{Help Desk}.

\subsection[Funzione di Help Desk]{funzione di help desk}
\label{sd-hd}
La funzionalità di \english{Help Desk}, presente in \ac{Information-Technology-Infrastructure-Library} v.2, fondamentalmente si focalizza sulle seguenti aree:

\begin{itemize}
\item{gestione di \english{software incident};}
\item{collaborazione con il processo di \ac{Change-Management}.}
\end{itemize}

Nella precedente versione del \english{framework} \ac{Information-Technology-Infrastructure-Library}, la versione 2, il \english{software} era il punto focale attorno a cui ruotavano tutti i processi.

Quindi la funzionalità di \english{Help Desk} si doveva solo concentrare nel ripristinare il \english{software}, a seguito di incidenti che lo rendessero meno usabile rispetto ai livelli pattuiti con il cliente/utente, oppure doveva coordinare le modifiche a quest'ultimo affinché il prodotto offerto fosse sempre allineato con le richieste.

\subsection[Funzione di Service Desk]{funzione di service desk}
\label{sd-sd}
Con la più recente versione del \english{framework}, la versione 3, il punto focale è diventato il \keyword{servizio}, quindi sono presenti nuove esigenze che un \english{Help Desk} non è in grado di soddisfare.

Data la chiara intenzione, esposta nel capitolato d'appalto, di rinnovare ed allinearsi a quanto di più moderno esista per offrire livelli di servizio che risultino essere il più \keyword{\glossaryPluralTerm{efficace}} ed \keyword{\glossaryPluralTerm{efficiente}} possibile l'ente proponente nel resto della proposta illustrerà una possibile implementazione della funzione di \english{Service Desk} in quanto a nostro avviso risponde pienamente alle richieste.

La restante parte del capitolo illustra gli scopi che tale funzionalità possiede, specificando benefici, risultati ed obiettivi. Successivamente vengono spiegati i differenti ruoli assunti dallo staff interno, la struttura che verrà adottata e chi sono gli utenti al quale si rivolge.

Infine, e non meno importate, verrà fornita una breve panoramica sugli strumenti che saranno adottati dallo staff del \english{Service Desk}.