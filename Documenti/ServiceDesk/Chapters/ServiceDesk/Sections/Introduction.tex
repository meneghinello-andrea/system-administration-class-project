%------------------------------------------------
%
% Introduction.tex 
%
% This section introduces the Service Desk
% function for the G. Pini hospital.
%------------------------------------------------
\section[Introduzione]{introduzione}
\label{sd-introduction}
Lo scopo principale di un \english{Service Desk} è quello di essere l'unico \ac{Service-Point-Of-Contact} tra gli utenti, che hanno necessità di assistenza/aiuto nell'utilizzo dei servizi \acs{Information-Technology}, e lo staff tecnico che risponde alle loro richieste.

Oltre ad essere una chiaro abilitatore di \english{business} (\keyword{\english{business enabler}}) esso attiva una crescita rigorosa dell'organizzazione che lo contiene. Inoltre le prestazioni di un \english{Service Desk} forniscono un'indicazione sul \keyword{livello generale di ``salute''} del dipartimento \acs{Information-Technology} presente all'interno della struttura ospedaliera.

\subsection[Benefici]{benefici}
\label{sd-introduction-benefit}
La funzione di \english{Service Desk} fornisce all'intero \entity{} i seguenti benefici:

\begin{itemize}
\item{incremento della percezione e soddisfazione del servizio utenti;}
\item{aumento dell'accessibilità ad assistenza/aiuto, della comunicazione e dell'informazione;}
\item{miglior qualità ed una veloce risposta delle richieste degli utenti;}
\item{incentiva il lavoro di gruppo e la comunicazione;}
\item{maggior attenzione ed un approccio proattivo alla fornitura di servizi;}
\item{miglior gestione e controllo dell'intera infrastruttura \acs{Information-Technology} presente;}
\item{incremento nell'uso delle risorse di supporto \acs{Information-Technology}.}
\end{itemize}

\subsection[Assicurare i risultati]{assicurare i risultati}
\label{sd-introduction-ensuring-results}
Nella realtà economica odierna spesso la riduzione dei costi è una necessità, ed i gruppi a supporto degli utenti sono generalmente i primi a subirli. E' perciò necessario assicurare che i servizi da loro offerti siano \keyword{chiaramente definiti} e \keyword{allineati} con i bisogni che il \english{business} richiede.

Al fine di introdurre e successivamente mantenere una funzionalità di \english{Service Desk} di successo nell'\entity{}, è essenziale che:

\begin{itemize}
\item{le necessità del \english{business} siano correttamente comprese;}
\item{le richieste degli utenti siano chiaramente comprese;}
\item{siano effettuati investimenti nella formazione dello staff tecnico;}
\item{obiettivi e risultati sino chiaramente definiti;}
\item{i livelli di servizio siano praticabili, accettati e regolarmente rivisiti.}
\end{itemize}

\subsection[Obiettivi]{obiettivi}
\label{sd-introduction-objectives}
L'obiettivo primario del \english{Service Desk}, che si vuole implementare nell'istituto, consiste nel \keyword{ripristino dei normali livelli di servizio agli utenti il più velocemente possibile}. 

In questo contesto il ``ripristino dei normali livelli di servizio'' è inteso nel modo più ampio possibile. Potrebbe comportare il ripristino a seguito di un guasto tecnico, oppure si potrebbe trattare di soddisfare una richiesta di servizio o ancora più semplicemente rispondere ad una domanda degli utenti. 

Deve essere fatto tutto ciò che risulti essere necessario al fine di consentire agli utenti di ritornare alle proprie mansioni il più velocemente possibile ed in modo soddisfacente.

Responsabilità specifiche comprendono:

\begin{itemize}
\item{il tracciamento tutti i dettagli riguardanti le richieste entranti;}
\item{la fornitura di una prima linea di supporto;}
\item{la risoluzione degli incidenti ``semplici'';}
\item{la ``chiusura'' di tutte le richieste che risultino essere soddisfatte;}
\item{la conduzione di sondaggi sulla soddisfazione percepita dagli utenti;}
\item{la comunicazione con gli utenti;}
\item{l'aggiornamento del \acf{Configuration-Management-DataBase}.}
\end{itemize}