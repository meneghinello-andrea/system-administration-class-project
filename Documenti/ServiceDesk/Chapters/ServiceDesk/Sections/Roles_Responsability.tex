%------------------------------------------------
%
% Roles-Responsability.tex 
%
% This section illustrates the roles and the
% responsability present in a service desk.
%------------------------------------------------
\section[Ruoli e responsabilità]{ruoli e responsabilità}
\label{sd-roles-responsabilities}
La chiave per un \english{Service Desk} \keyword{efficace} ed \keyword{efficiente} è quella di garantire che vi sia una chiara responsabilità e che i ruoli siano chiaramente definiti in modo da portare a termine la pratica di servizio.

Data la taglia dell'\entity{}, come esso è strutturato e l'esistenza di \english{patners} esterni, si rende necessaria l'adozione dei seguenti ruoli all'interno della funzionalità di \english{Service Desk}:

\begin{itemize}
\item{Service Desk Manager}
\item{Service Desk Supervisor}
\item{Service Desk Analyst}
\end{itemize}

Per ogni ruolo è necessario che siano ben definite, redatte ed approvate le descrizioni delle mansioni affinché le responsabilità specifiche siano note.

\subsection[Service Desk Manager]{service desk manager}
\label{sd-sd-manger}
Il \english{Service Desk Manager} prende la responsabilità per le seguenti attività:

\begin{itemize}
\item{gestione di tutte le attività del \english{Service Desk} inclusa la supervisione;}
\item{agire come ulteriore punto di \english{escalation} per i supervisori;}
\item{riportare alla sezione di \english{Service Strategy} ogni richiesta che può significativamente impattare il \english{business} dell'\entity{};}
\item{partecipare alle sedute del \ac{Change-Advisory-Board};}
\item{assumersi la responsabilità globale per la gestione degli incidenti e la richiesta di realizzazione del \english{Service Desk}. Ciò può essere esteso a qualsiasi altra attività assunta dal \english{Service Desk}.}
\end{itemize}

I requisiti richiesti affinché un membro del personale dell'\entity{} possa essere candidato alla carica di \english{Service Desk Manager} sono:

\begin{itemize}
\item{possedere conoscenze tecniche;}
\item{possedere capacità di gestione del personale;}
\item{possedere una visione globale dell'intero sistema \acs{Information-Technology} su cui l'istituto ospedaliero si appoggia per svolgere i propri compiti;}
\item{possedere abbastanza esperienza per comprendere velocemente le possibili cause di un incidente;}
\item{fornire la propria reperibilità a fronte di gravi situazioni anche al di fuori del normale orario di operatività del \english{Service Desk}.}
\end{itemize}

\subsection[Service Desk Supervisor]{service desk supervisor}
\label{sd-sd-supervisor}
La precedente suddivisione in aree delle risorse (vedi Sezione \ref{sd-resources-categories}) porta alla definizione del ruolo di \english{Service Desk Supervisor} che possiede le seguenti responsabilità per la propria area di competenza:

\begin{itemize}
\item{stabilire i turni di lavoro del proprio staff affinché il livello di servizio sia garantito nei momenti in cui i servizi dell'area in questione sono maggiormente utilizzati;}
\item{intraprendere attività di \ac{Human-Resources} in base alle esigenze;}
\item{produzione di statistiche e report per l'area di competenza;}
\item{rappresentare l'area di competenza durante le riunioni con il \english{Service Desk Manager};}
\item{organizzare la formazione del personale della propria area;}
\item{collegamento con il processo di \ac{Change-Management};}
\item{effettuare dei \english{briefings} con lo staff in merito a cambiamenti o sviluppi che possono influenzare il volume delle richieste presso il \english{Service Desk};}
\item{assistere gli analisti durante la prima linea di supporto quando i carichi di lavoro sono elevati, o dove la richiesta richiede maggior esperienza.}
\end{itemize}

\attribute{nota}: Un singolo individuo può essere il responsabile di più aree di competenza.

I requisiti richiesti affinché un membro del personale dell'\entity{} possa essere candidato alla carica di \english{Service Desk Supervisor} sono:

\begin{itemize}
\item{possedere buone conoscenze tecniche dell'area di competenza;}
\item{possedere capacità di gestione del personale;}
\item{fornire la propria disponibilità a fronte di gravi emergenze anche al di fuori dell'orario di operatività del \english{Service Desk}.}
\end{itemize}

I \english{Service Desk Supervisor} fanno riferimento alla figura del \english{Service Desk Manager} in caso di necessità.

\subsection[Service Desk Analyst]{service desk analyst}
\label{sd-sd-analyst}
Il ruolo primario di un \english{Service Desk Analyst} è quello di fornire un supporto di primo livello attraverso la ricezione delle chiamate e successiva gestione degli incidenti che ne conseguono oppure la gestione delle richieste di servizio oppure il monitoraggio dell'ambiente \acs{Information-Technology}.

Le sue attività avvengono all'interno dei processi di \acf{Incident-Management}, \acf{Request-Fulfillment} ed \acf{Event-Management} in linea con gli obiettivi del \english{Service Desk}.