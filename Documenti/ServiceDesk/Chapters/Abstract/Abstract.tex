%------------------------------------------------
%
% Abstract.tex 
%
% This section introduce the document.
%------------------------------------------------
\chapter[Abstract]{abstract}
\label{abs}
All'interno di una realtà aziendale complessa, come l'ente oggetto di questo capitolato, è doveroso istituire ed istruire dei processi e delle funzioni di controllo che mirano a garantire che i servizi offerti agli utenti siano il più \keyword{efficaci} ed \keyword{efficienti} possibile.

Un possibile metodo per ottenere degli ottimi livelli di servizi offerti e allo stesso tempo avere un \keyword{miglioramento permanente e continuo}, degli stessi, consiste nell'adottare un \english{\glossaryTerm{framework}} quale, per esempio, \ac{ITIL} per la gestione del proprio dipartimento \acs{IT}.

Lo scopo che questo documento si prefigge di raggiungere consiste nella progettazione e la conseguente implementazione di una particolare funzionalità presente nel \english{framework}, la funzionalità di \keyword{\english{Service Desk}}. Verranno inoltre illustrati alcuni dei processi, che direttamente, ha lo scopo di amministrare.

Essi sono: 

\begin{itemize}
\item{\acf{EM}}
\item{\acf{IM}}
\item{\acf{RF}}
\end{itemize}

%------------------------------------------------
%
% Structure.tex 
%
% This section describe the structure of the
% document.
%------------------------------------------------
\section[Struttura del documento]{struttura del documento}
\label{abs-document-structure}
Il seguente documento è stato suddiviso nei seguenti capitoli:

\begin{itemize}
\item{\keyword{Capitolo 1}: funzionalità di \english{Service Desk}}
\item{\keyword{Capitolo 2}: definizione dei processi}
\item{\keyword{Capitolo 3}: \ac{Service-Level-Requirements} della funzione di \english{Service Desk}}
\item{\keyword{Capitolo 4}: \ac{Service-Level-Agreement} della funzione di \english{Service Desk}}
\item{\keyword{Capitolo 5}: piano di \english{roll out} del nuovo \english{Service Desk}}
\end{itemize}

%------------------------------------------------
%
% Recipient.tex 
%
% This section show the recipient of this
% document.
%------------------------------------------------
\section{A chi è rivolto}
\label{abs-recipient}
Il presente documento è rivolto ai membri del consiglio di amministrazione dell'Ospedale Gaetano Pini di Milano. Alla C.A. dei dottori:

\begin{itemize}
\item{Dott. Amedeo Tropiano -- Direttore Generale}
\item{Dott.ssa Loredana Maspes -- Direttore Amministrativo}
\item{Dott. Nunzio Angelo Buccino -- Direttore Sanitario}
\end{itemize}

%------------------------------------------------
%
% Hightlighting.tex 
%
% This section describe the hightlightining
% used in the document.
%------------------------------------------------
\section[Convenzioni tipografiche]{convenzioni tipografiche}
\label{abs-hightlighting}
Nel seguente documento si sono introdotte le seguenti convenzioni tipografiche al fine di consentire una lettura scorrevole dello stesso.

Esse sono:

\begin{itemize}
\item{\textbf{grassetto} per evidenziare i termini più importanti presenti nel paragrafo;}
\item{\textit{corsivo} per evidenziare i termini in lingua inglese;}
\item{\underline{sottolineato} per la prima occorrenza di termini presenti nel glossario finale.}
\end{itemize}