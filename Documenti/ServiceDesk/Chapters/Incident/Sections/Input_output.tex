%------------------------------------------------
%
% Input_output.tex 
%
% This section introduces the incident management
% process.
%------------------------------------------------
\section[Input e output del processo]{input e output del processo}
\label{im-io}
In questa sezione vengono illustrati gli \english{input} e gli \english{output} del processo, specificando da chi sono ricevuti e a chi sono rivolti.

\subsection[Input del processo]{input del processo}
\label{im-io-input}
In Tabella \ref{im-io-input-table} sono elencati gli \english{input} del processo e da chi sono ricevuti.

\begin{center}
\begin{longtable}{| p{6cm} | p{7cm} |}
\caption{Input del processo}
\label{im-io-input-table}\\
\hline
\multicolumn{1}{| c |}{\textbf{Input}} & \multicolumn{1}{| c |}{\textbf{Da}}\\
\hline
\endfirsthead
\hline
\multicolumn{1}{| c |}{\textbf{Input}} & \multicolumn{1}{| c |}{\textbf{Da}}\\
\hline
\endhead
Incidente (scritto) & Utenti\\
\hline
Soluzione a problemi & Processo di \acf{Problem-Management}\\
\hline
Risposta a richieste di cambiamento & Processo di \acf{Change-Management}\\
\hline
\end{longtable}
\end{center}

Gli \english{input} che esso riceve avvengono tramite quello che viene definito come \english{\glossarySingolarTerm{ticket}}, che nasce generalmente con un incidente, prosegue con un problema ed infine diventa un errore.

Viene manipolato da personale differente in momenti diversi, per maggiori dettagli vedere le attività di processo (Sezione REF).

I campi e gli attributi che compongono un \english{ticket} sono illustrati nelle Tabelle \ref{im-io-input-ticket-common-table}, \ref{im-io-input-ticket-upgrade-table}, \ref{im-io-input-ticket-attachment-table}, \ref{im-io-input-ticket-resolution-table} e \ref{im-io-input-ticket-history-table} in cui valgono le seguenti definizioni:

\begin{itemize}
\item{\attribute{sola lettura}: nessun dato può essere inserito in questo campo;}
\item{\attribute{auto generato}: i dati di questo campo sono inseriti automaticamente dal sistema;}
\item{\attribute{casella di controllo}: una casella che se cliccata abilita la visione della sezione associata (\english{checkbox});}
\item{\attribute{collegamento}: il campo prevede la presenza di un controllo cliccabile che conduce l'operatore nella banca dati in cui può selezionare il valore corretto per popolare parte dei campi del \english{ticket};}
\item{\attribute{risposta}: campo in cui l'utente può inserire testo a piacere (una sola riga);}
\item{\attribute{risposta aperta}: campo in cui l'utente può inserire testo a piacere (più righe);}
\item{\attribute{data}: campo in cui è presente un dato in formato data;}
\item{\attribute{lista}: consente all'utente di scegliere un valore da una lista di valori predefinita.}
\end{itemize}

\clearpage{}

\begin{center}
\begin{longtable}{| p{3cm} | p{6.5cm} | p{3cm} |}
\caption{Informazioni di base di un \english{ticket}}
\label{im-io-input-ticket-common-table}\\
\hline
\multicolumn{1}{| c |}{\textbf{Campo}} & \multicolumn{1}{| c |}{\textbf{Descrizione}} & \multicolumn{1}{| c |}{\textbf{Tipo di campo}}\\
\endfirsthead
\hline
\multicolumn{1}{| c |}{\textbf{Campo}} & \multicolumn{1}{| c |}{\textbf{Descrizione}} & \multicolumn{1}{| c |}{\textbf{Tipo di campo}}\\
\endhead
\hline
\multicolumn{3}{| c |}{\textit{sezione di sistema}}\\
\hline
\english{Ticket} ID & Numero identificativo incrementale del \english{ticket}. Per evitare il problema dell'\glossarySingolarTerm{overflow} dovuto ad un alto numero di richieste in ingresso il campo assume il seguente formato \attribute{yyyymmdd-xxxxx}, dove il prefisso rappresenta il giorno in cui il \english{ticket} è giunto al processo tramite la funzione di \english{Service Desk} mentre il suffisso rappresenta un numero incrementale che si azzera ogni giorno solare. Questa scelta consente di raggiungere un tetto massimo di \num{99999} \english{ticket} giornalieri. & \attribute{sola lettura} -- \attribute{auto generato}\\
\hline
\multicolumn{3}{| c |}{\textit{sezione utente}}\\
\hline
Seleziona & Abilita il responsabile del \english{ticket} a selezionare, dal \english{database}, i dati l'utente che ha necessità di assistenza. & \attribute{collegamento}\\
\hline
Cognome & Cognome dell'utente che ha necessità di assistenza. & \attribute{sola lettura} -- \attribute{auto generato}\\
\hline
Nome & Nome dell'utente che ha necessità di assistenza. & \attribute{sola lettura} -- \attribute{auto generato}\\
\hline
Identificativo & Codice identificativo dell'utente che ha necessità di assistenza. & \attribute{sola lettura} -- \attribute{auto generato}\\
\hline
E-mail & Indirizzo e-mail con cui è possibile contattare l'utente della richiesta. & \attribute{sola lettura} -- \attribute{auto generato}\\
\hline
Telefono & Numero di telefono con cui è possibile contattare l'utente della richiesta. & \attribute{sola lettura} -- \attribute{auto generato}\\
\hline
Interno & Numero dell'interno dell'utente della richiesta. & \attribute{sola lettura} -- \attribute{auto generato}\\
\hline
FAX & Numero di FAX dell'utente oppure del dipartimento in cui risiede. & \attribute{sola lettura} -- \attribute{auto generato}\\
\hline
Numero stanza & Numero dell'ufficio dell'utente. & \attribute{sola lettura} -- \attribute{auto generato}\\
\hline
Piano & Numero del piano in cui si trova l'ufficio dell'utente. & \attribute{sola lettura} -- \attribute{auto generato}\\
\hline
Utente critico & Questa casella di controllo, se abilitata, indica che l'utente della richiesta ha una maggior priorità sugli altri e necessità di un servizio più tempestivo. & \attribute{casella di controllo} -- \attribute{auto generato}\\
\hline
\multicolumn{3}{| c |}{\textit{sezione referente}}\\
\hline
Seleziona & Abilita il responsabile del \english{ticket} a selezionare, dal \english{database}, i dati del referente dell'utente che ha comunicato il disservizio. & \attribute{collegamento}\\
\hline
Cognome & Nome del referente dell'utente che ha riportato la richiesta. & \attribute{sola lettura} -- \attribute{auto generato}\\
\hline
Nome & Cognome del referente dell'utente che ha riportato la richiesta. & \attribute{sola lettura} -- \attribute{auto generato}\\
\hline
Telefono & Numero di telefono del referente che ha riportato la richiesta. & \attribute{sola lettura} -- \attribute{auto generato}\\
\hline
Interno & Numero dell'interno del referente che ha riportato la richiesta. & \attribute{sola lettura} -- \attribute{auto generato}\\
\hline
FAX & Numero di FAX dell'utente che ha riportato la richiesta o del dipartimento in cui risiede. & \attribute{sola lettura} -- \attribute{auto generato}\\
\hline
Numero stanza & Numero dell'ufficio dell'utente. & \attribute{sola lettura} -- \attribute{auto generato}\\
\hline
Piano & Numero del piano in cui si trova l'ufficio dell'utente. & \attribute{sola lettura} -- \attribute{auto generato}\\
\hline
\multicolumn{3}{| c |}{\textit{sezione tecnica}}\\
\hline
Stato & Stato in cui si trova il \english{ticket}. Vedi Sezione \ref{prc-incident-status}. & \attribute{lista}\\
\hline
Responsabile & Responsabile della gestione del \english{ticket}. Inizialmente è popolato dall'operatore che risponde alla richiesta, tuttavia, può essere sempre modificato se il suo responsabile, per qualsiasi motivo, cambia. Il responsabile può essere solo un membro attivo del \english{Service Desk}. & \attribute{collegamento}\\
\hline
Tipo & Specifica la tipologia di \english{ticket}. (vedi Sezione \ref{prc-incident-category} per maggiori dettagli) & \attribute{lista}\\
\hline
Categoria & Specifica la categoria presso cui si colloca il \english{ticket}. (vedi Sezione \ref{im-management-categorization} per maggiori dettagli). & \attribute{lista}\\
\hline
Prodotto & Specifica il prodotto che è affetto dall'anomalia. & \attribute{lista}\\
\hline
Impatto & Specifica l'impatto che ha per il \english{business} l'incidente. L'impatto è definito su una scala da \num{1} a \num{5} dove uno significa ``non grave'' mentre cinque significa ``grave''. & \attribute{lista}\\
\hline
Urgenza & Specifica l'urgenza nella risoluzione dell'incidente. & \attribute{lista}\\
\hline
Priorità & La priorità è definita dallo sforzo/impegno necessario alla risoluzione del \english{ticket}. & \attribute{lista}\\
\hline
\ac{Service-Level-Agreement} & Accordo sul livello di servizio concordato. & \attribute{collegamento}\\
\hline
\multicolumn{3}{| c |}{\textit{sezione \ac{Configuration-Item}}}\\
\hline
Identificativo & L'identificativo del \ac{Configuration-Item} coinvolto nell'incidente. & \attribute{collegamento}\\
\hline
Tipo & Indica la tipologia di \ac{Configuration-Item} (\english{hardware}, \english{software}, stampante, ecc.) & \attribute{sola lettura} -- \attribute{auto generato}\\
\hline
Modello & Modello del \ac{Configuration-Item} & \attribute{sola lettura} -- \attribute{auto generato}\\
\hline
\multicolumn{3}{| c |}{\textit{sezione livello di supporto di grado superiore}}\\
\hline
Gruppo & Indice la seconda linea di supporto a cui il \english{ticket} può essere riassegnato in caso di necessità. & \attribute{collegamento}\\
\hline
Responsabile & Membro del gruppo della seconda linea di supporto responsabile del \english{ticket}. & \attribute{colleagmento}\\
\hline
Telefono & Numero di telefono per contattare il responsabile del \english{ticket} nella seconda linea di supporto. & \attribute{sola lettura} -- \attribute{auto generato}\\
\hline
\end{longtable}
\end{center}

I campi riportati in Tabella \ref{im-io-input-ticket-upgrade-table} rappresentano le informazioni riguardanti l'incidente che sono disponibili a seguito di una analisi tecnica.

Se le cause scatenanti dell'incidente a cui si riferisce il \english{ticket} sono multiple, allora i seguenti campi sono ripetuti tante volte quante sono le cause.

Tramite i seguenti campi è possibile tracciare la cronistoria dell'incidente.

\begin{center}
\begin{longtable}{| p{3cm} | p{6.5cm} | p{3cm} |}
\caption{Informazioni di aggiornamento del \english{ticket}}
\label{im-io-input-ticket-upgrade-table}\\
\hline
\multicolumn{1}{| c |}{\textbf{Campo}} & \multicolumn{1}{| c |}{\textbf{Descrizione}} & \multicolumn{1}{| c |}{\textbf{Tipo di campo}}\\
\endfirsthead
\hline
\multicolumn{1}{| c |}{\textbf{Campo}} & \multicolumn{1}{| c |}{\textbf{Descrizione}} & \multicolumn{1}{| c |}{\textbf{Tipo di campo}}\\
\endhead
\hline
\multicolumn{3}{| c |}{\textit{sezione cause}}\\
\hline
ID Causa & Il codice identificativo della causa dell'incidente. Può cambiare durante il ciclo di vita del \english{ticket}. & \attribute{lista}\\
\hline
Descrizione breve & Fornisce una breve descrizione delle cause dell'incidente. & \attribute{risposta}\\
\hline
Descrizione & Fornisce una descrizione completa delle cause dell'incidente. & \attribute{risposta aperta}\\
\hline
\multicolumn{3}{| c |}{\textit{sezione di sistema}}\\
\hline
Aggiornato il & Data di aggiornamento del \english{ticket}. & \attribute{data} -- \attribute{auto generato}\\
\hline
\end{longtable}
\end{center}

Il \english{software} a sostegno delle attività di processo (vedi Sezione \ref{sd-tools}) fornisce degli strumenti che consentono ai membri dello staff di associare altri \english{ticket} a quello in lavorazione. Risulta utile quando il \english{ticket} attuale riguarda un incidente con somiglianze ad uno già presente nella banca dati.

L'associazione è fatta in automatico e quando l'associazione avviene vedremo aggiunti campi riportati in Tabella \ref{im-io-input-ticket-attachment-table}.

\begin{center}
\begin{longtable}{| p{3cm} | p{6.5cm} | p{3cm} |}
\caption{Informazioni da altri documenti interni}
\label{im-io-input-ticket-attachment-table}\\
\hline
\multicolumn{1}{| c |}{\textbf{Campo}} & \multicolumn{1}{| c |}{\textbf{Descrizione}} & \multicolumn{1}{| c |}{\textbf{Tipo di campo}}\\
\endfirsthead
\hline
\multicolumn{1}{| c |}{\textbf{Campo}} & \multicolumn{1}{| c |}{\textbf{Descrizione}} & \multicolumn{1}{| c |}{\textbf{Tipo di campo}}\\
\endhead
\hline
\multicolumn{3}{| c |}{\textit{sezione \english{ticket} collegato}}\\
\hline
ID incidente & Numero identificativo del \english{ticket} di riferimento. & \attribute{sola lettura} -- \attribute{auto generato}\\
\hline
Aperto il & Data di apertura del \english{ticket} di riferimento. & \attribute{data} -- \attribute{sola lettura} -- \attribute{auto generato}\\
\hline
Stato & Stato del \english{ticket} di riferimento. & \attribute{sola lettura} -- \attribute{auto generato}\\
\hline
Tipo & Tipologia del \english{ticket} di riferimento. & \attribute{sola lettura} -- \attribute{auto generato}\\
\hline
Categoria & Categoria del \english{ticket} di riferimento. & \attribute{sola lettura} -- \attribute{auto generato}\\
\hline
Descrizione breve & Descrizione breve dell'incidente a cui si riferisce il \english{ticket}. & \attribute{sola lettura} -- \attribute{auto generato}\\
\hline
\multicolumn{3}{| c |}{\textit{sezione \english{problem} collegato}}\\
\hline
ID problema & Numero identificativo del problema di riferimento all'incidente. & \attribute{sola lettura} -- \attribute{auto generato}\\
\hline
Aperto il & Data di apertura del problema di riferimento. & \attribute{data} -- \attribute{sola lettura} -- \attribute{auto generato}\\
\hline
Stato & Stato del problema di riferimento. & \attribute{sola lettura} -- \attribute{auto generato}\\
\hline
Categoria & Categoria in cui risiede il problema di riferimento. & \attribute{sola lettura} -- \attribute{auto generato}\\
\hline
Descrizione breve & Breve descrizione del problema di riferimento. & \attribute{sola lettura} -- \attribute{auto generato}\\
\hline
\multicolumn{3}{| c |}{\textit{sezione \english{known error} collegato}}\\
\hline
ID errore & Numero identificativo dell'errore collegato. & \attribute{sola lettura} -- \attribute{auto generato}\\
\hline
Aperto il & Data di apertura dell'errore di riferimento. & \attribute{data} -- \attribute{sola lettura} -- \attribute{auto generato}\\
\hline
Stato & Stato dell'errore di riferimento. & \attribute{sola lettura} -- \attribute{auto generato}\\
\hline
Categoria & Categoria dell'errore di riferimento. & \attribute{sola lettura} -- \attribute{auto generato}\\
\hline
Descrizione breve & Breve descrizione dell'errore di riferimento & \attribute{sola lettura} -- \attribute{auto generato}\\
\hline
\multicolumn{3}{| c |}{\textit{sezione \ac{Request-For-Change} collegata}}\\
\hline
\english{change} ID & Numero identificativo della \ac{Request-For-Change} di riferimento. & \attribute{sola lettura} -- \attribute{auto generato}\\
\hline
Categoria & Categoria in cui risiede la \ac{Request-For-Change} di riferimento. & \attribute{sola lettura} -- \attribute{auto generato}\\
\hline
Fase & Fase in cui risiede la \ac{Request-For-Change} di riferimento. & \attribute{sola lettura} -- \attribute{auto generato}\\
\hline
\english{Asset} & \english{Asset} a cui si riferisce la \ac{Request-For-Change} di riferimento. & \attribute{sola lettura} -- \attribute{auto generato}\\
\hline
Descrizione breve & Breve descrizione della \ac{Request-For-Change} di riferimento. & \attribute{sola lettura} -- \attribute{auto generato}\\
\hline
Inizio pianificato il & Data di inizio pianificata per l'attuazione della modifica. & \attribute{data} -- \attribute{sola lettura} -- \attribute{auto generato}\\
\hline
Fine pianificata il & Data di fine pianificata per l'attuazione della modifica. & \attribute{data} -- \attribute{sola lettura} -- \attribute{auto generato}\\
\hline
\end{longtable}
\end{center}

Quando il \english{ticket} viene risolto i campi riportati in Tabella \ref{im-io-input-ticket-resolution-table} sono popolati dal suo responsabile.

\begin{center}
\begin{longtable}{| p{3cm} | p{6.5cm} | p{3cm} |}
\caption{Informazioni di risoluzione dell'incidente}
\label{im-io-input-ticket-resolution-table}\\
\hline
\multicolumn{1}{| c |}{\textbf{Campo}} & \multicolumn{1}{| c |}{\textbf{Descrizione}} & \multicolumn{1}{| c |}{\textbf{Tipo di campo}}\\
\endfirsthead
\hline
\multicolumn{1}{| c |}{\textbf{Campo}} & \multicolumn{1}{| c |}{\textbf{Descrizione}} & \multicolumn{1}{| c |}{\textbf{Tipo di campo}}\\
\endhead
\hline
ID risoluzione & Numero identificativo della risoluzione dell'incidente. & \attribute{lista}\\
\hline
Descrizione breve & Descrizione breve della risoluzione. & \attribute{risposta}\\
\hline
Descrizione & Descrizione completa della risoluzione dell'incidente. & \attribute{risposta aperta}\\
\hline
\end{longtable}
\end{center}

Il \english{software} propone una visione ridotta del \english{ticket} in cui vediamo la cronistoria. I campi visualizzati sono elencati in Tabella \ref{im-io-input-ticket-history-table}.

\begin{center}
\begin{longtable}{| p{3cm} | p{6.5cm} | p{3cm} |}
\caption{Storico del \english{ticket}}
\label{im-io-input-ticket-history-table}\\
\hline
\multicolumn{1}{| c |}{\textbf{Campo}} & \multicolumn{1}{| c |}{\textbf{Descrizione}} & \multicolumn{1}{| c |}{\textbf{Tipo di campo}}\\
\endfirsthead
\hline
\multicolumn{1}{| c |}{\textbf{Campo}} & \multicolumn{1}{| c |}{\textbf{Descrizione}} & \multicolumn{1}{| c |}{\textbf{Tipo di campo}}\\
\endhead
\hline
Aperto da & Cognome e nome dell'utente/cliente che ha aperto la richiesta di assistenza. & \attribute{sola lettura} -- \attribute{auto generato}\\
\hline
Aperto il & Data di apertura della richiesta da parte dell'utente/cliente. & \attribute{data} -- \attribute{sola lettura} -- \attribute{auto generato}\\
\hline
Aggiornato il & Data di aggiornamento della richiesta di assistenza. & \attribute{data} -- \attribute{sola lettura} -- \attribute{auto generato}\\
\hline
Risolto da & Cognome e nome del membro dello staff che ha risolto l'incidente. & \attribute{sola lettura} -- \attribute{auto generato}\\
\hline
Risolto il & Data di risoluzione dell'incidente. & \attribute{data} -- \attribute{sola lettura} -- \attribute{auto generato}\\
\hline
Chiuso da & Cognome e nome del membro dello staff che ha chiuso la richiesta dell'utente/cliente. & \attribute{sola lettura} -- \attribute{auto generato}\\
\hline
Chiuso il & Data di chiusura e archiviazione della richiesta. & \attribute{data} -- \attribute{sola lettura} -- \attribute{auto generato}\\
\hline
\end{longtable}
\end{center}

\subsection[Output del processo]{output del processo}
\label{im-io-output}
In Tabella \ref{im-io-output-table} sono elencati gli \english{output} del processo e a chi sono rivolti.

\begin{center}
\begin{longtable}{| p{6cm} | p{7cm} |}
\caption{output del processo}
\label{im-io-output-table}\\
\hline
\multicolumn{1}{| c |}{\textbf{Input}} & \multicolumn{1}{| c |}{\textbf{Da}}\\
\hline
\endfirsthead
\hline
\multicolumn{1}{| c |}{\textbf{Input}} & \multicolumn{1}{| c |}{\textbf{Da}}\\
\hline
\endhead
Notifiche agli utenti quanto il \english{ticket} cambia di stato. & Utenti\\
\hline
\end{longtable}
\end{center}