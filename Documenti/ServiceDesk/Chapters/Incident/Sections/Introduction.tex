%------------------------------------------------
%
% Introduction.tex 
%
% This section introduces the incident management
% process.
%------------------------------------------------
\section[Introduzione]{introduzione}
\label{im-introduction}
Questo capitolo descrive il processo di \acf{Incident-Management} per la funzione di \english{Service Desk} dell'\entity{}.

Il processo fornisce un metodo coerente che gli utenti devono seguire in caso di disservizi durante la normale operatività all'interno dell'\entity{}.

\subsection[Scopo principale]{scopo principale}
\label{im-introduction-scope}
Lo scopo principale del processo di \ac{Incident-Management} è quello di \keyword{ripristinare il funzionamento normale dei servizi il più rapidamente possibile} e \keyword{ridurre al minimo l'impatto negativo} sulle operazioni di \english{business}, garantendo così che i migliori livelli possibili di \keyword{qualità del servizio} e \keyword{disponibilità} siano mantenuti.

\attribute{nota}: attraverso la dicitura ``normale funzionamento del servizio'' si intende quando un servizio sta operando all'interno dei limiti imposti dallo \ac{Service-Level-Agreement}.

Il processo viene applicato a tutti gli incidenti, sia tecnici che utente (vedi Sezione \ref{im-introduction-definition}), che possono avvenire durante l'utilizzo dei servizi \acs{Information-Technology} messi a disposizione degli utenti.

Vengono escluse da questo processo:

\begin{itemize}
\item{le richieste di servizio: sono gestite dal processo di \acf{Request-Fulfillment} (vedi Capitolo \ref{rf});}
\item{la ricerca delle cause scatenanti: è una delle attività che compongono il processo di \acf{Problem-Management} (non trattato in questo documento).}
\end{itemize}

\subsection[Definizione di processo]{definizione di processo}
\label{im-introduction-definition}
Questo processo include qualsiasi evento che innesca comportamenti anomali, o che potrebbe interrompere un servizio. Include inoltre gli eventi che vengono comunicati direttamente dagli utenti attraverso la funzione di \english{Service Desk} oppure attraverso un'interfaccia interrogabile dal processo di \ac{Event-Management} verso lo strumento a supporto delle attività di processo.

\subsection[Obiettivi]{obiettivi}
\label{im-introduction-objectives}
Instaziando il processo si vogliono raggiungere i seguenti \keyword{obiettivi}:

\begin{itemize}
\item{gli incidenti devono essere propriamente registrati;}
\item{gli incidenti devono essere propriamente instradati;}
\item{lo stato di ogni incidente deve essere segnalato con precisione;}
\item{la coda degli incidenti non ancora risolti deve essere visibile e segnalata;}
\item{gli incidenti devono essere propriamente corredati di priorità e gestiti nella sequenza corretta;}
\item{la risoluzione fornita deve soddisfare i requisiti dello \ac{Service-Level-Agreement}.}
\end{itemize}

\subsection[Definizioni]{definizioni}
\label{im-introduction-definitions}
In questa sezione seguono brevi definizioni e precisazioni riguardo i termini utilizzati nel contesto del processo di \ac{Incident-Management}. Questi termini sono qui esplicitati al fine di garantire chiarezza nella comprensione dei contenuti nelle sezioni seguenti.

\subsubsection{impatto}
L'impatto è determinato attraverso il numero di utenti o funzioni che sono affette dall'incidente. Il proponente definisce tre gradi di impatto per l'\entity{} illustrati in Tabella \ref{im-introduction-definition-impact-table}.

\begin{center}
\begin{longtable}{| p{4cm} | p{6cm} | p{2cm} |}
\caption{Gradi di impatto}
\label{im-introduction-definition-impact-table}\\
\hline
\multicolumn{1}{| c |}{\textbf{Definizione}} & \multicolumn{1}{| c |}{\textbf{Descrizione}} & \multicolumn{1}{| c |}{\textbf{Grado}}\\
\hline
\endfirsthead
\hline
\multicolumn{1}{| c |}{\textbf{Definizione}} & \multicolumn{1}{| c |}{\textbf{Descrizione}} & \multicolumn{1}{| c |}{\textbf{Grado}}\\
\hline
\endhead
\attribute{basso} & L'incidente affligge un massimo di due o tre utenti. Il servizio è degradato, ma ancora operativo nei termini dello \ac{Service-Level-Agreement}. & \multicolumn{1}{| c |}{3}\\
\hline
\attribute{medio} & Molteplici utenti in uno stesso reparto sono affetti dall'incidente. Il servizio è degradato e ancora funzionante, ma non operativo nelle specifiche dello \ac{Service-Level-Agreement}. & \multicolumn{1}{| c |}{2}\\
\hline
\attribute{alto} & Tutti gli utenti di un servizio sono affetti dall'incidente. Il servizio non è più operativo. & \multicolumn{1}{| c |}{1}\\
\hline
\end{longtable}
\end{center}

L'impatto di un incidente viene utilizzato nel calcolo della sua priorità.

\subsubsection{incidente}
Un incidente è una interruzione non pianificata di un servizio \acs{Information-Technology} o una riduzione della qualità di un servizio \acs{Information-Technology}.

E' considerato incidente anche il fallimento di un qualsiasi prodotto, \english{software} o \english{hardware}, utilizzato nel supporto di un sistema che non è ancora interessato dall'anomalia (ad esempio il guasto di un componente di una configurazione ad elevata disponibilità ridondante è un incidente anche se non interrompe il servizio offerto).

Si noti che un difetto di produzione o progettazione non è un incidente. Se il prodotto funziona come progettato, ma la progettazione risulta essere errata la correzione deve assumere la forma di un richiesta di servizio che ha un ciclo di vita differente (vedi Capitolo REF).

\subsubsection{incidente utente}
Questa tipologia contiene gli incidenti che sono maggiormente incontrati dagli utenti durante il normale utilizzo dei servizi \acs{Information-Technology}. Possono riguardare sia le applicazioni che l'\english{hardware} che essi utilizzano.

\subsubsection{incidente tecnico}
Questa tipologia contiene incidenti che possono accadere senza che l'utente ne sia consapevole. Potrebbe esserci, per esempio, una risposta più lenta della rete o su una specifica \english{workstation} ma, finché il degrado è graduale l'utente potrebbe non notarlo.

\subsubsection{incident repository}
L'\english{incident repository} è un \english{database} logico contenente le informazioni rilevanti su tutti gli incidenti riscontrati, risolti o ancora pendenti.

\subsubsection{priorità}
La priorità di un incidente viene determinata utilizzando una combinazione di impatto e severità. Per una spiegazione completa fare riferimento alla Sezione \ref{im-management-priority}.

\subsubsection{risposta}
Tempo trascorso tra il momento in cui l'incidente è segnalato e quando viene assegnato ad un membro dello staff del \english{Service Desk} per la sua risoluzione.

\subsubsection{risoluzione}
Il servizio è riportato in uno stato coerente con le specifiche dello \ac{Service-Level-Agreement}. In alcuni casi però, il servizio pur ritornando ad essere operativo presenta un livello di servizio inferiore a quanto stabilito ma consente comunque all'utente di tornare ad essere operativo finché il problema viene ricercato e risolto.

\subsubsection{Gravità}
La gravità di un incidente viene determinata sulla base di quanto l'utente è limitato nello svolgere le proprie mansioni. Il proponente definisce tre gradi di gravità per l'\entity{} illustrati in Tabella \ref{im-introduction-definition-severity-table}.

\begin{center}
\begin{longtable}{| p{4cm} | p{6cm} | p{2cm} |}
\caption{Gradi di gravità}
\label{im-introduction-definition-severity-table}\\
\hline
\multicolumn{1}{| c |}{\textbf{Definizione}} & \multicolumn{1}{| c |}{\textbf{Descrizione}} & \multicolumn{1}{| c |}{\textbf{Grado}}\\
\hline
\endfirsthead
\hline
\multicolumn{1}{| c |}{\textbf{Definizione}} & \multicolumn{1}{| c |}{\textbf{Descrizione}} & \multicolumn{1}{| c |}{\textbf{Grado}}\\
\hline
\endhead
\attribute{basso} & L'incidente impedisce all'utente di svolgere parte delle mansioni. & \multicolumn{1}{| c |}{3}\\
\hline
\attribute{medio} & L'incidente impedisce all'utente di svolgere funzioni sensibili in un momento critico. & \multicolumn{1}{| c |}{2}\\
\hline
\attribute{alto} & L'incidente affligge un intero servizio o la maggior parte di esso. & \multicolumn{1}{| c |}{1}\\
\hline
\end{longtable}
\end{center}

La gravità di un incidente viene utilizzata per determinare la priorità dell'incidente.