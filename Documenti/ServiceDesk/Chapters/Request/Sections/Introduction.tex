%------------------------------------------------
%
% Introduction.tex 
%
% This section introduces the request fulfillment
% process.
%------------------------------------------------
\section[Introduzione]{introduzione}
\label{im-introduction}
Questo capitolo descrive il processo di \acf{Request-Fulfillment} per la funzione di \english{Service Desk} dell'\entity{}.

Il processo fornisce un metodo coerente che gli utenti devono seguire in caso di richieste di servizio compatibili con il \english{Service Catalogue} esposto.

\subsection[Scopo principale]{scopo principale}
\label{im-introduction-scope}
Lo scopo principale del processo \acf{Request-Fulfillment} è quello di soddisfare le richieste, provenienti dagli utenti, che però non sono disservizi.

Vengono escluse da questo processo:

\begin{itemize}
\item{gli incidenti che sono gestiti dal processi di \ac{Incident-Management} (vedi Capitolo \ref{im};)}
\item{modifiche a sistemi come risultato del processo di \ac{Problem-Management}.}
\end{itemize}

\subsection[Obiettivi]{obiettivi}
\label{im-introduction-objectives}
Instaziando il processo si vogliono raggiungere i seguenti \keyword{obiettivi}:

\begin{itemize}
\item{le \english{service request} devono essere propriamente registrate;}
\item{le \english{service request} devono essere propriamente instradate;}
\item{lo stato di ogni \english{service request} deve essere segnalato con precisione;}
\item{la coda delle \english{service request} non ancora risolte deve essere visibile e segnalata;}
\item{le \english{service request} devono essere debitamente dimensionate in base a costo/beneficio;}
\item{le \english{service request} devono essere propriamente corredate di priorità e gestite nella sequenza corretta;}
\item{le \english{service request} fornite devono soddisfare i requisiti dell'utente;}
\item{lo sforzo non è sprecato su richieste non approvate.}
\end{itemize}

\subsection[Definizioni]{definizioni}
\label{im-introduction-definitions}
In questa sezione seguono brevi definizioni e precisazioni riguardo i termini utilizzati nel contesto del processo di \ac{Request-Fullfillment}. Questi termini sono qui esplicitati al fine di garantire chiarezza nella comprensione dei contenuti nelle sezioni seguenti.

\subsubsection{service request}
Una \english{Service Request} è una richiesta iniziata dall'utente. La richiesta è specifica su servizi presenti nel \english{Service Catalogue}.

Il processo differenzia due tipologie di richieste:

\begin{itemize}
\item{\textbf{\ac{Assistant-Service-Request}}: richiesta di un sostegno continuo di \english{routine} al fine di mantenere attiva la produzione dell'utente. Per essere considerata una richiesta di \english{routine} deve presentare le seguenti caratteristiche:}
\begin{itemize}
\item{requisiti riconosciuti;}
\item{soluzione conosciuta;}
\item{costo conosciuto;}
\item{rischio conosciuto;}
\item{nessun impatto nello \ac{Service-Level-Agreement};}
\item{richiesta pre approvata.}
\end{itemize}
\item{\textbf{\ac{Enhancement-Change-Request}}: richiesta per la modifica o aggiunta di nuove funzionalità ad un servizio esistente.}
\end{itemize}

\subsubsection{carenza}
Una carenza si ha quando il servizio fornito non incontra, pienamente o parzialmente le richieste espresse dall'utente oppure quanto definito nello \ac{Service-Level-Agreement}.

\subsubsection{service request standard}
Le richieste standard corrispondono alle \ac{Assistant-Service-Request} poiché possiedono aspettative predefinite, standard di qualità e tempestività di servizio pre approvati.

\subsubsection{request repository}
Il \english{request repository} è un \english{database} logico in cui vengono inserite tutte le informazioni rilevanti sulle \english{Service Request} compiute e pendenti.

%\subsection{