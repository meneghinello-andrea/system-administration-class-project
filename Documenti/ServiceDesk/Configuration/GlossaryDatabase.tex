%------------------------------------------------
%
% GlossaryDatabase.tex
%
% This file contains the terms that are present
% in the glossary of the document.
%------------------------------------------------
\newglossaryentry{configuration-item}
{
	name		= {\acf{Configuration-Item}},
	plural		= {configuration items},
	description = {Un elemento della configurazione (\english{configuration item}) è un qualsiasi componente che necessita di essere gestito per poter erogare un servizio \acs{Information-Technology}. Le informazioni su ogni \acs{Configuration-Item} vengono registrate in un \acf{Configuration-Record} all'interno del \acf{Configuration-Management-System} ed aggiornate per tutto il suo ciclo di vita dal \acf{Configuration-Management}. I \ac{Configuration-Item} sono sotto il controllo del processo di \acs{Change-Management}. Tipicamente fra i \ac{Configuration-Item} includiamo i servizi \acs{Information-Technology}, \english{hardware}, \english{software}, edifici, persone e documentazione formale quale la documentazione di processo e di \acf{Service-Level-Agreement}}
}

\newglossaryentry{deming}
{
	name		= {ciclo di Deming},
	plural		= {ciclo di Deming},
	description = {Il ciclo di Deming (ciclo \acf{Plan-Do-Check-Act}) è un modello studiato per il miglioramento continuo della qualità in un'ottica a lungo raggio. Serve per promuovere una cultura della qualità che è tesa al miglioramento continuo dei processi e all'utilizzo ottimale delle risorse. Questo strumento parte dall'assunto che per il raggiungimento del massimo della qualità sia necessaria la costante interazione tra ricerca, progettazione, test, produzione e vendita. Per migliorare la qualità e soddisfare il cliente, le quattro fasi devono ruotare costantemente, tenendo come criterio principale la qualità. La sequenza logica dei quattro punti ripetuti è la seguente: P = pianificazione; D = esecuzione; C = test e controllo; A = attuazione per rendere definitivi i miglioramenti}
}

\newglossaryentry{efficace}
{
	name		= {efficace},
	plural		= {efficaci},
	description = {Un servizio che produce pienamente l'effetto richiesto o desiderato}
}

\newglossaryentry{efficiente}
{
	name		= {efficiente},
	plural		= {efficienti},
	description = {Raggiungimento dell'obiettivo preposto nel modo migliore possibile ossia usando al meglio le risorse a propria disposizione}
}

\newglossaryentry{errore}
{
	name		= {errore},
	plural		= {errori},
	description = {In ambiente \ac{Information-Technology-Infrastructure-Library} viene definito errore un problema la cui soluzione è ormai nota}
}

\newglossaryentry{framework}
{
	name		= {framework},
	plural		= {frameworks},
	description = {In informatica, specialmente nello sviluppo \english{software}, è un'architettura logica di supporto su cui un \english{software}/servizio può essere progettato e realizzato spesso facilitandone lo sviluppo}
}

\newglossaryentry{incidente}
{
	name		= {incidente},
	plural		= {incidenti},
	description	= {Un incidente (\english{incident}) è un qualsiasi evento che non fa parte dell'operatività \english{standard} del servizio e che causa una riduzione o un'interruzione del servizio offerto}
}

\newglossaryentry{overflow}
{
	name		= {overflow},
	plural		= {overflow},
	description = {Condizione che accade quando un calcolo produce un risultato che è più grande di quello che può essere memorizzato}
}

\newglossaryentry{patch}
{
	name		= {patch},
	plural		= {patches},
	description = {La produzione di \english{software}, è usualmente soggetta ad errori di scrittura del codice sorgente e malfunzionamenti, chiamati \english{bug}, che vengono scoperti solo dopo il rilascio del \english{software} stesso. Nel suo significato primario, \english{patch} (pezza), è un termine inglese che indica un file eseguibile creato per risolvere uno specifico errore di programmazione, che impedisce il corretto funzionamento di un programma o di un sistema. Tali \english{files} vengono rilasciati dagli stessi produttori, nell'attesa di una nuova versione del \english{software} stesso}
}

\newglossaryentry{release}
{
	name		= {release},
	plural		= {releases},
	description = {Nell'ambito dello sviluppo di un servizio una \english{release} è una specifica versione dello stesso resa disponibile agli utenti finali. E' univocamente identificata in modo che sia possibile distinguerla dalle precedenti. Convenzionalmente si suddividono in \english{release} maggiori (\english{Major Release}) quando le differenze dalla precedente versione riguardano sostanziali evoluzioni delle funzionalità offerte e \english{release} minori (\english{Minor Release}) quando le differenze riguardano principalmente correzioni di malfunzionamenti}
}

\newglossaryentry{rollout}
{
	name		= {roll out},
	plural		= {roll out},
	description = {Nel campo dei sistemi informativi il termine \english{roll out} è comunemente impiegato per identificare il processo con cui un servizio viene attivato a partire da una istanza iniziale. Durante il \english{roll out} del servizio si trovano le fasi progettuali di \english{training}, di collaudo e di messa in esercizio}
}

\newglossaryentry{ticket}
{
	name		= {ticket},
	plural		= {tickets},
	description = {In ambito \ac{Information-Technology-Infrastructure-Library} ed in particolare nella sotto sezione del \english{framework} che contiene la funzione di \english{Service Desk} un \english{ticket} è il gergo tecnico che si riferisce ad una qualsiasi richiesta in ingresso alla funzione di \english{Service Desk}}
}

\newglossaryentry{workaround}
{
	name		= {workaround},
	plural		= {workarounds},
	description	= {Si tratta di una correzione temporanea ad un incidente oppure una sequenza di azioni alternative a quella che produce l'incidente che però rende utilizzabile, anche se in versione ridotta, il servizio}
}